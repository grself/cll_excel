%*****************************************
\chapter{Statistics}\label{ch08:statistics}
%*****************************************

Intro

\section{x}

\begin{center}
	\begin{objbox}{Learning Objectives}
		\begin{itemize}
			\setlength{\itemsep}{0pt}
			\setlength{\parskip}{0pt}
			\setlength{\parsep}{0pt}
			
			\item one

		\end{itemize}
	\end{objbox}
\end{center}

Lorem

\begin{center}
	\begin{tkwbox}{Key Take-Aways}
		\textbf{x}
		\\
		\begin{itemize}
			\setlength{\itemsep}{0pt}
			\setlength{\parskip}{0pt}
			\setlength{\parsep}{0pt}
			
			\item one
			
		\end{itemize}
	\end{tkwbox}
\end{center}

\section{Preparing to Print}

\begin{center}
	\begin{objbox}{Learning Objectives}
		\begin{itemize}
			\setlength{\itemsep}{0pt}
			\setlength{\parskip}{0pt}
			\setlength{\parsep}{0pt}
			
			\item Review each worksheet in a workbook in Print Preview.
			\item Modify worksheets as needed to professionally print data and charts.
			
		\end{itemize}
	\end{objbox}
\end{center}

Lorem 

\begin{center}
	\begin{tkwbox}{Key Take-Aways}
		\textbf{Preparing to Print}
		\\
		\begin{itemize}
			\setlength{\itemsep}{0pt}
			\setlength{\parskip}{0pt}
			\setlength{\parsep}{0pt}
			
			\item Formatting for print is an essential part of creating a workbook.
			
		\end{itemize}
	\end{tkwbox}
\end{center}

\section{Chapter Practice}

\subsection{Project Team Analysis}

\textit{Data file: PR7-Data.csv}


\section{Scored Assessment}

\subsection{Employee Analysis}

\textit{Data file: SC7-Data.csv}

