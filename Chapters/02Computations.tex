%*****************************************
\chapter{Mathematical Computations}\label{ch02:mathematical_computations}
%*****************************************

Perhaps the most valuable feature of Excel is its ability to produce mathematical outputs using the data in a workbook. This chapter reviews several mathematical outputs that you can produce in Excel through the construction of formulas and functions. The chapter begins with the construction of formulas for basic and complex mathematical computations. The second section reviews statistical functions, such as SUM, AVERAGE, MIN, and MAX, which can be applied to a range of cells. The last section of the chapter addresses functions used to calculate mortgage and lease payments as well as the valuation of investments. This chapter also shows how you can use data from multiple worksheets to construct formulas and functions. These skills will be demonstrated in the context of a personal cash budget, which is a vital tool for managing your money for long-term financial security. The personal budget objective will also provide you with several opportunities to demonstrate Excel's what-if scenario capabilities, which highlight how formulas and functions automatically produce new outputs when one or more inputs are changed.

\section{Formulas}

\begin{center}
	\begin{objbox}{Learning Objectives}
		\begin{itemize}
			\setlength{\itemsep}{0pt}
			\setlength{\parskip}{0pt}
			\setlength{\parsep}{0pt}
			
			\item Learn how to create basic formulas.
			\item Understand relative referencing when copying and pasting formulas.
			\item Work with complex formulas by controlling the order of mathematical operations.
			\item Understand formula auditing tools.

		\end{itemize}
	\end{objbox}
\end{center}

This section reviews the fundamental skills for entering formulas into an Excel worksheet. The objective used for this chapter is the construction of a personal cash budget. Most financial advisors recommend that all households construct and maintain a personal budget to achieve and maintain strong financial health. Organizing and maintaining a personal budget is a skill you can practice at any point in your life. Whether you are managing your expenses during college or maintaining the finances of a family of four, a personal budget can be a vital tool when making financial decisions. Excel can make managing your money a fun and rewarding exercise.

Figure \ref{02:fig01} shows the completed workbook that will be demonstrated in this chapter. Notice that this workbook contains four worksheets. The first worksheet, \textbf{Budget Summary}, contains formulas that utilize or reference the data in the other three worksheets. As a result, the \textbf{Budget Summary} worksheet serves as an overview of the data that was entered and calculated in the other three worksheets of the workbook.

\begin{figure}[H]
	\centering
	\includegraphics[width=\maxwidth{.95\linewidth}]{gfx/ch02_fig01}
	\caption{Completed Personal Budget Workbook}
	\label{02:fig01}
\end{figure}

\subsection{Creating a Basic Formula}

\textit{Download Data File: CH2 Data}

Formulas are used to calculate a variety of mathematical outputs in Excel and can be used to create virtually any custom calculation required for your objective. Furthermore, when constructing a formula in Excel, you use cell locations that, when added to a formula, become cell references. This means that Excel uses, or references, the number entered into the cell location when calculating a mathematical output. As a result, when the numbers in the cell references are changed, Excel automatically produces a new output. This is what gives Excel the ability to create a variety of what-if scenarios, which will be explained later in the chapter.

To demonstrate the construction of a basic formula, we will begin working on the \textbf{Budget Detail} worksheet in the Personal Budget workbook, which is shown in Figure \ref{02:fig02}. To complete this worksheet, we will add several formulas and functions. Table \ref{02:tab01} provides definitions for each of the spend categories listed in the range \textsf{A3:A11}. When you develop a personal budget, these categories are defined on the basis of how you spend your money. It is likely that every person could have different categories or define the same categories differently. Therefore, it is important to review the definitions in Table \ref{02:tab01} to understand how we are defining these categories before proceeding.

\begin{figure}[H]
	\centering
	\includegraphics[width=\maxwidth{.95\linewidth}]{gfx/ch02_fig02}
	\caption{Budget Detail Worksheet}
	\label{02:fig02}
\end{figure}

{\small
	\begin{longtable}{p{0.85in}p{2.8in}}
		\textbf{Category} & \textbf{Definition} \endhead
		\hline \\
		Household\newline Utilities & Money spent on electricity, heat, and water and on cable, phone, and Internet access\\
		Food & Money spent on groceries, toiletries, and related items\\
		Gasoline & Money spent on fuel for automobiles\\
		Clothes & Money spent on clothes, shoes, and accessories\\
		Insurance & Money spent on homeowner's or automobile insurance\\
		Taxes & Money spent on school and property taxes (this example of the personal budget assumes that we own property)\\
		Entertainment & Money spent on entertainment, including dining out, movie and theater tickets, parties, and so on\\
		Vacation & Money spent on vacations\\
		Miscellaneous & Includes any other spending categories, such as textbooks, software, journals, school or work supplies, and so on\\
		\hline
		\caption{Spend Category Definitions}
		\label{02:tab01}
	\end{longtable}
}

The first formula that we will add to the \textbf{Budget Detail} worksheet will calculate the Monthly Spend values. The formula will be constructed so that it takes the values in the Annual Spend column and divides them by $ 12 $. This will show how much money will be spent per month for each of the categories listed in Column \textsf{A}. The following explains how this formula is created.

\begin{enumerate}
	\item Open the Data file named \textbf{CH2 Data} and use the File/Save As command to save it with the new name \textbf{CH2 Personal Budget}.
	\item Click the \textbf{Budget Detail} worksheet tab to open the worksheet.
	\item Click cell \textsf{C3}.
	\item Type an equal sign $ = $. When the first character entered into a cell location is an equal sign, it signals Excel to perform a calculation or produce a logical output.
	\item Type \textsf{D3}. This adds \textsf{D3} to the formula, which is now a cell reference. Excel will use whatever value is entered into cell \textsf{D3} to produce an output.
	\item Type the slash symbol $ / $. This is the symbol for division in Excel. As shown in Table \ref{02:tab02} the mathematical operators in Excel are slightly different from those found on a typical calculator.
	\item Type the number $ 12 $. This divides the value in cell \textsf{D3} by $ 12 $. In this formula, a number, or constant, is used instead of a cell reference because it will not change. In other words, there will always be $ 12 $ months in a year.
	\item Press the \keystroke{Enter} key.
\end{enumerate}

\begin{longtable}{p{0.85in}p{2.8in}}
	\textbf{Symbol} & \textbf{Operation} \endhead
	\hline \\
	$ + $ & Addition\\
	$ - $ & Subtraction\\
	$ / $ & Division\\
	$ * $ & Multiplication\\
	$ \wedge $ & Power/Exponent\\
	\hline
	\caption{Excel Mathematical Operators}
	\label{02:tab02}
\end{longtable}

Figure \ref{02:fig03} shows how the formula appears in cell \textsf{C3} before you press the \keystroke{Enter} key. Figure \ref{02:fig04} shows the output of the formula after you press the \keystroke{Enter} key. The Monthly Spend for Household Utilities is \$250 because the formula is taking the Annual Spend in cell \textsf{D3} and dividing it by 12. If the value in cell \textsf{D3} is changed, the formula automatically produces a new output. We are calculating the spend per month for each category because people often get paid and are billed for these items on a monthly basis. This formula allows you to compare your monthly income to your monthly bills to determine whether you have enough income to pay these expenses.

\begin{figure}[H]
	\centering
	\includegraphics[width=\maxwidth{.95\linewidth}]{gfx/ch02_fig03}
	\caption{Adding a Formula to a Worksheet}
	\label{02:fig03}
\end{figure}

\begin{figure}[H]
	\centering
	\includegraphics[width=\maxwidth{.95\linewidth}]{gfx/ch02_fig04}
	\caption{Formula Output for Monthly Spend}
	\label{02:fig04}
\end{figure}

\begin{center}
	\begin{infobox}{Why?}
		\textbf{Use Cell References}
		\\
		\\
		Cell references enable Excel to dynamically produce new outputs when one or more inputs in the referenced cells are changed. Cell references also allow you to trace how outputs are being calculated in a formula. As a result, you should never use a calculator to determine a mathematical output and type it into the cell location of a worksheet. Doing so eliminates Excel's cell-referencing benefits as well as your ability to trace a formula to determine how outputs are being produced.
	\end{infobox}
\end{center}

\subsection{Relative References (Copying and Pasting Formulas)}

Once a formula is typed into a worksheet, it can be copied and pasted to other cell locations. For example, Figure \ref{02:fig04} shows the output of the formula that was entered into cell \textsf{C3}. However, this calculation needs to be performed for the rest of the cell locations in Column \textsf{C}. Since we used the \textsf{D3} cell reference in the formula, Excel automatically adjusts that cell reference when the formula is copied and pasted into the rest of the cell locations in the column. This is called relative referencing and is demonstrated as follows.

\begin{enumerate}
	\item Click cell \textsf{C3}.
	\item Place the mouse pointer over the Auto Fill Handle.
	\item When the mouse pointer turns from a white block plus sign to a black plus sign, click and drag down to cell \textsf{C11}. This pastes the formula into the range \textsf{C4:C11}.
	\item Double click cell \textsf{C6}. Notice that the cell reference in the formula is automatically changed to \textsf{D6}.
	\item Press the \keystroke{Enter} key.
\end{enumerate}

Figure \ref{02:fig05} shows the outputs added to the rest of the cell locations in the Monthly Spend column. For each row, the formula takes the value in the Annual Spend column and divides it by 12. You will also see that cell \textsf{D6} has been double clicked to show the formula. Notice that Excel automatically changed the original cell reference of \textsf{D3} to \textsf{D6}. This is the result of relative referencing, which means Excel automatically adjusts a cell reference relative to its original location when it is pasted into new cell locations. In this example, the formula was pasted into eight cell locations below the original cell location. As a result, Excel increased the row number of the original cell reference by a value of one for each row it was pasted into.

\begin{figure}[H]
	\centering
	\includegraphics[width=\maxwidth{.95\linewidth}]{gfx/ch02_fig05}
	\caption{Relative Reference Example}
	\label{02:fig05}
\end{figure}

\begin{center}
	\begin{infobox}{Why?}
		\textbf{Use Universal Constants}
		\\
		\\
		If you are using constants, or numerical values, in an Excel formula, they should be universal constants that do not change, such as the number of days in a week, weeks in a year, and so on. Do not type the values that exist in cell locations into an Excel formula. This will eliminate Excel's cell-referencing benefits, which means if the value in the cell location you are using in a formula is changed, Excel will not be able to produce a new output.
	\end{infobox}
\end{center}

\subsection{Creating Complex Formulas (Controlling the Order of Operations)}

The next formula to be added to the Personal Budget workbook is the percent change over last year. This formula determines the difference between the values in the LY (Last Year) Spend column and shows the difference in terms of a percentage. This requires that the order of mathematical operations be controlled to get an accurate result. Table 2.3 shows the standard order of operations for a typical formula. To change the order of operations shown in the table, we use parentheses to process certain mathematical calculations first. This formula is added to the worksheet as follows.

\begin{enumerate}
	\item Click cell \textsf{F3} in the Budget Detail worksheet.
	\item Type an equal sign $ = $.
	\item Type an open parenthesis $ ( $.
	\item Click cell \textsf{D3}. This will add a cell reference to cell \textsf{D3} to the formula. When building formulas, you can click cell locations instead of typing them.
	\item Type a minus sign $ - $.
	\item Click cell \textsf{E3} to add this cell reference to the formula.
	\item Type a closing parenthesis $ ) $.
	\item Type the slash $ / $ symbol for division.
	\item Click cell \textsf{E3}. This completes the formula that will calculate the percent change of last year's actual spent dollars vs. this year's budgeted spend dollars (see Figure \ref{02:fig06}).
	\item Press the \keystroke{Enter} key.
	\item Click cell \textsf{F3} to activate it.
	\item Place the mouse pointer over the Auto Fill Handle.
	\item When the mouse pointer turns from a white block plus sign to a black plus sign, click and drag down to cell \textsf{F11}. This pastes the formula into the range \textsf{F4:F11}.
\end{enumerate}

\begin{figure}[H]
	\centering
	\includegraphics[width=\maxwidth{.95\linewidth}]{gfx/ch02_fig06}
	\caption{Adding the Percent Change Formula}
	\label{02:fig06}
\end{figure}

\begin{longtable}{p{0.85in}p{2.8in}}
	\textbf{Symbol} & \textbf{Operation} \endhead
	\hline \\
	$ () $ & Override Standard Order: Any mathematical computations placed in parentheses are performed first and override the standard order of operations. If there are layers of parentheses used in a formula, Excel computes the innermost parentheses first and the outermost parentheses last.\\
	$ \wedge $ & First: Excel executes any exponential computations first.\\
	$ * $ or $ / $ & Second: Excel performs any multiplication or division computations second. When there are multiple instances of these 	computations in a formula, they are executed in order from left to right.\\
	$ + $ or $ - $ & Third: Excel performs any addition or subtraction computations third. When there are multiple instances of these 	computations in a formula, they are executed in order from left to right.\\
	\hline
	\caption{Standard Order of Mathematical Operations}
	\label{02:tab03}
\end{longtable}

\begin{center}
	\begin{infobox}{Why?}
		\textbf{Use Relative Referencing}
		\\
		\\
		Relative referencing is a convenient feature in Excel. When you use cell references in a formula, Excel automatically adjusts the cell references when the formula is pasted into new cell locations. If this feature were not available, you would have to manually retype the formula when you want the same calculation applied to other cell locations in a column or row.
	\end{infobox}
\end{center}

Figure \ref{02:fig06} shows the formula that was added to the \textbf{Budget Detail} worksheet to calculate the percent change in spending. The parentheses were added to this formula to control the order of operations. Any mathematical computations placed in parentheses are executed first before the standard order of mathematical operations (see Table \ref{02:tab03}). In this case, if parentheses were not used, Excel would produce an erroneous result for this worksheet.

Figure \ref{02:fig07} shows the result of the percent change formula if the parentheses are removed. The formula produces a result of a $ 299900\% $ increase. Since there is no change between the LY spend and the budget Annual Spend, the result should be 0\%. However, without the parentheses, Excel is following the standard order of operations. This means the value in cell \textsf{E3} will be divided by \textsf{E3} first ($ 3,000 / 3,000 $), which is $ 1 $. Then, the value of $ 1 $ will be subtracted from the value in cell \textsf{D3} ($ 3,000 - 1 $), which is $ 2,999 $. Since cell \textsf{F3} is formatted as a percentage, Excel expresses the output as an increase of $ 299900\% $.

\begin{figure}[H]
	\centering
	\includegraphics[width=\maxwidth{.95\linewidth}]{gfx/ch02_fig07}
	\caption{Removing the Parentheses from the Percent Change Formula}
	\label{02:fig07}
\end{figure}

\begin{center}
	\begin{sklbox}{Skill Refresher}
		\textbf{Formulas}
		\\
		\begin{itemize}
			\setlength{\itemsep}{0pt}
			\setlength{\parskip}{0pt}
			\setlength{\parsep}{0pt}
			
			\item Type an equal sign $ = $.
			\item Click or type a cell location. If using constants, type a number.
			\item Type a mathematical operator.
			\item Click or type a cell location. If using constants, type a number.
			\item Use parentheses where necessary to control the order of operations.
			\item Press the \keystroke{Enter} key.

			\item one
			\item two
			
		\end{itemize}
	\end{sklbox}
\end{center}

\subsection{Auditing Formulas}

Excel provides a few tools that you can use to review the formulas entered into a worksheet. For example, instead of showing the outputs for the formulas used in a worksheet, you can have Excel show the formula as it was entered in the cell locations. This is demonstrated as follows.

\begin{enumerate}
	\item With the Budget Detail worksheet open, click the Formulas tab of the Ribbon.
	\item Click the Show Formulas button in the Formula Auditing group of commands. This displays the formulas in the worksheet instead of showing the mathematical outputs.
	\item Click the Show Formulas button again. The worksheet returns to showing the output of the
formulas.
\end{enumerate}

Figure \ref{02:fig08} shows the Budget Detail worksheet after activating the Show Formulas command in the Formulas tab of the Ribbon. As shown in the figure, this command allows you to view and check all the formulas in a worksheet without having to click each cell individually. After activating this command, the column widths in your worksheet increase significantly. The column widths were adjusted for the worksheet shown in Figure \ref{02:fig08} so all columns can be seen. The column widths return to their previous width when the Show Formulas command is deactivated.

\begin{figure}[H]
	\centering
	\includegraphics[width=\maxwidth{.95\linewidth}]{gfx/ch02_fig08}
	\caption{Show Formulas Command}
	\label{02:fig08}
\end{figure}

\begin{center}
	\begin{infobox}{Integrity Check}
		\textbf{Does the Output of Your Formula Make Sense?}
		\\
		\\
		It is important to note that the accuracy of the output produced by a formula depends on how it is constructed. Therefore, always check the result of your formula to see whether it makes sense with data in your worksheet. As shown in Figure \ref{02:fig07}, a poorly constructed formula can give you an inaccurate result. In other words, you can see that there is no change between the Annual Spend and LY Spend for Household Utilities. Therefore, the result of the formula should be 0\%. However, since the parentheses were removed in this case, the formula is clearly producing an erroneous result.
	\end{infobox}
\end{center}

\begin{center}
	\begin{sklbox}{Skill Refresher}
		\textbf{Show Formulas}
		\\
		\begin{itemize}
			\setlength{\itemsep}{0pt}
			\setlength{\parskip}{0pt}
			\setlength{\parsep}{0pt}
			
			\item Click the Formulas tab on the Ribbon.
			\item Click the Show Formulas button in the Formula Auditing group of commands.
			\item Click the Show Formulas button again to show formula outputs.
			
		\end{itemize}
	\end{sklbox}
\end{center}

\begin{center}
	\begin{shtcutbox}{Keyboard Shortcuts}
		\textbf{Show Formulas}
		\\
		\begin{itemize}
			\setlength{\itemsep}{0pt}
			\setlength{\parskip}{0pt}
			\setlength{\parsep}{0pt}
			
			\item Hold down the \keystroke{Ctrl} key while pressing the accent symbol `.
			
		\end{itemize}
	\end{shtcutbox}
\end{center}

Two other tools in the Formula Auditing group of commands are the Trace Precedents and Trace Dependents commands. These commands are used to trace the cell references used in a formula. A precedent cell is a cell whose value is used in other cells. The Trace Precedents command shows an arrow to indicate the cells or ranges (precedents) which affect the active cell’s value. A dependent cell is a cell whose value depends on the values of other cells in the workbook. The Trace Dependents command shows where any given cell is referenced in a formula. The following is a demonstration of these commands.

\begin{enumerate}
	\item Click cell \textsf{D3} in the \textbf{Budget Detail} worksheet.
	\item Click the Trace Dependents button in the Formula Auditing group of commands in the Formulas tab of the Ribbon. A double blue arrow appears, pointing to cell locations \textsf{C3} and \textsf{F3} (see Figure \ref{02:fig09}). This indicates that cell \textsf{D3} is referenced in formulas that are entered in cells \textsf{C3} and \textsf{F3}.
	\item Click the Remove Arrows command in the Formula Auditing group of commands in the Formulas tab of the Ribbon. This removes the Trace Dependents arrow.
	\item Click cell \textsf{F3} in the \textbf{Budget Detail} worksheet.
	\item Click the Trace Precedents button in the Formula Auditing group of commands in the Formulas tab of the Ribbon. A blue arrow running through cells \textsf{D3} and \textsf{E3} and pointing to cell \textsf{F3} appears (see Figure \ref{02:fig10}). This indicates that cells \textsf{D3} and \textsf{E3} are references in a formula entered in cell \textsf{F3}.
	\item Click the Remove Arrows command in the Formula Auditing group of commands in the Formulas tab of the Ribbon. This removes the Trace Precedents arrow.
	\item Save the \textbf{Ch2 Personal Budget} file.
\end{enumerate}

Figure \ref{02:fig09} shows the Trace Dependents arrow on the Budget Detail worksheet. The blue dot represents the activated cell. The arrows indicate where the cell is referenced in formulas.

\begin{figure}[H]
	\centering
	\includegraphics[width=\maxwidth{.95\linewidth}]{gfx/ch02_fig09}
	\caption{Trace Dependents Example}
	\label{02:fig09}
\end{figure}

Figure \ref{02:fig10} shows the Trace Precedents arrow on the Budget Detail worksheet. The blue dots on this arrow indicate the cells that are referenced in the formula contained in the activated cell. The arrow is pointing to the activated cell location that contains the formula.

\begin{figure}[H]
	\centering
	\includegraphics[width=\maxwidth{.95\linewidth}]{gfx/ch02_fig10}
	\caption{Trace Precedents Example}
	\label{02:fig10}
\end{figure}

\begin{center}
	\begin{sklbox}{Skill Refresher}
		\textbf{Trace Dependents}
		\\
		\begin{itemize}
			\setlength{\itemsep}{0pt}
			\setlength{\parskip}{0pt}
			\setlength{\parsep}{0pt}
			
			\item Click a cell location that contains a number or formula.
			\item Click the Formulas tab on the Ribbon.
			\item Click the Trace Dependents button in the Formula Auditing group of commands.
			\item Use the arrow(s) to determine where the cell is referenced in formulas and functions.
			\item Click the Remove Arrows button to remove the arrows from the worksheet.
			
		\end{itemize}
	\end{sklbox}
\end{center}

\begin{center}
	\begin{sklbox}{Skill Refresher}
		\textbf{Trace Precedents}
		\\
		\begin{itemize}
			\setlength{\itemsep}{0pt}
			\setlength{\parskip}{0pt}
			\setlength{\parsep}{0pt}
			
			\item Click a cell location that contains a formula or function.
			\item Click the Formulas tab on the Ribbon.
			\item Click the Trace Precedents button in the Formula Auditing group of commands.
			\item Use the dot(s) along the line to determine what cells are referenced in the formula or function.
			\item Click the Remove Arrows button to remove the line with the dots.
			
		\end{itemize}
	\end{sklbox}
\end{center}

\begin{center}
	\begin{tkwbox}{Key Take-Aways}
		\textbf{Formulas}
		\\
		\begin{itemize}
			\setlength{\itemsep}{0pt}
			\setlength{\parskip}{0pt}
			\setlength{\parsep}{0pt}
			
			\item Mathematical computations are conducted through formulas and functions.
			\item An equal sign $ = $ precedes all formulas and functions.
			\item Formulas and functions must be created with cell references to conduct what-if scenarios where mathematical outputs are recalculated when one or more inputs are changed.
			\item Mathematical operators on a typical calculator are different from those used in Excel. Table \ref{02:tab02}, ``Excel Mathematical Operators,'' lists Excel mathematical operators.
			\item When using numerical values in formulas and functions, only use universal constants that do not change, such as days in a week, months in a year, and so on.
			\item Relative referencing automatically adjusts the cell references in formulas and functions when they are pasted into new locations on a worksheet. This eliminates the need to retype formulas and functions when they are needed in multiple rows or columns on a worksheet.
			\item Parentheses must be used to control the order of operations when necessary for complex formulas.
			\item Formula auditing tools such as Trace Dependents, Trace Precedents, and Show Formulas should be used to check the integrity of formulas that have been entered into a worksheet.
			
		\end{itemize}
	\end{tkwbox}
\end{center}

\section{Statistical Functions}\label{ch02:statistical_functions}

\begin{center}
	\begin{objbox}{Learning Objectives}
		\begin{itemize}
			\setlength{\itemsep}{0pt}
			\setlength{\parskip}{0pt}
			\setlength{\parsep}{0pt}
			
			\item Use the SUM function to calculate totals.
			\item Use absolute references to calculate percent of totals.
			\item Use the COUNT function to count cell locations with numerical values.
			\item Use the AVERAGE function to calculate the arithmetic mean.
			\item Use the MAX and MIN functions to find the highest and lowest values in a range of cells.
			\item Learn how to copy and paste formulas without formats applied to a cell location.
			\item Learn how to set a multiple level sort sequence for data sets that have duplicate values or outputs.
			
 		\end{itemize}
	\end{objbox}
\end{center}

In addition to formulas, another way to conduct mathematical computations in Excel is through functions. Statistical functions apply a mathematical process to a group of cells in a worksheet. For example, the SUM function is used to add the values contained in a range of cells. A list of commonly used statistical functions is shown in Table \ref{02:tab04}. Functions are more efficient than formulas when you are applying a mathematical process to a group of cells. If you use a formula to add the values in a range of cells, you would have to add each cell location to the formula one at a time. This can be very time-consuming if you have to add the values in a few hundred cell locations. However, when you use a function, you can highlight all the cells that contain values you wish to sum in just one step. This section demonstrates a variety of statistical functions that we will add to the Personal Budget workbook. In addition to demonstrating functions, this section also reviews percent of total calculations and the use of absolute references.

{\small
	\begin{longtable}{p{0.75in}p{3.0in}}
		\textbf{Function} & \textbf{Output} \endhead
		\hline \\
		ABS & The absolute value of a number\\
		AVERAGE & The average or arithmetic mean for a group of numbers\\
		COUNT & The number of cell locations in a range that contain a numeric character\\
		COUNTA & The number of cell locations in a range that contain a text or numeric character\\
		MAX & The highest numeric value in a group of numbers\\
		MEDIAN & The middle number in a group of numbers (half the numbers in the group are higher than the median and half the numbers in the group are lower than the median)\\
		MIN & The lowest numeric value in a group of numbers\\
		MODE & The number that appears most frequently in a group of numbers\\
		PRODUCT & The result of multiplying all the values in a range of cell locations\\
		SQRT & The positive square root of a number\\
		STDEV.S & The standard deviation for a group of numbers based on a sample\\
		SUM & The total of all numeric values in a group\\
		\caption{Commonly Used Statistical Functions}
		\label{02:tab04}
	\end{longtable}
}

The following discusses a few of the more commonly-used statistical functions.

\subsection{The Sum Function}

The SUM function is used when you need to calculate totals for a range of cells or a group of selected cells on a worksheet. With regard to the Budget Detail worksheet, we will use the SUM function to calculate the totals in row 12. It is important to note that there are several methods for adding a function to a worksheet, which will be demonstrated throughout the remainder of this chapter. The following illustrates how a function can be added to a worksheet by typing it into a cell location.

\begin{enumerate}
	\item Click the Budget Detail worksheet tab to open the worksheet.
	\item Click cell \textsf{C12}.
	\item Type an equal sign $ = $.
	\item Type the function name SUM.
	\item Type an open parenthesis $ ( $.
	\item Click cell \textsf{C3} and drag down to cell \textsf{C11}. This places the range \textsf{C3:C11} into the function.
	\item Type a closing parenthesis $ ) $.
	\item Press the \keystroke{Enter} key. The function calculates the total for the Monthly Spend column, which is \$1,496.
\end{enumerate}

Figure \ref{02:fig11} shows the appearance of the SUM function added to the Budget Detail worksheet before pressing the \keystroke{Enter} key.

\begin{figure}[H]
	\centering
	\includegraphics[width=\maxwidth{.95\linewidth}]{gfx/ch02_fig11}
	\caption{Adding the SUM Function to the Budget Detail Worksheet}
	\label{02:fig11}
\end{figure}

As shown in Figure \ref{02:fig11}, the SUM function was added to cell \textsf{C12}. However, this function is also needed to calculate the totals in the Annual Spend and LY Spend columns. The function can be copied and pasted into these cell locations because of relative referencing. Relative referencing serves the same purpose for functions as it does for formulas. The following demonstrates how the total row is completed.

\begin{enumerate}
	\item Click cell \textsf{C12} in the Budget Detail worksheet.
	\item Click the Copy button in the Home tab of the Ribbon.
	\item Highlight cells \textsf{D12} and \textsf{E12}.
	\item Click the Paste button in the Home tab of the Ribbon. This pastes the SUM function into cells \textsf{D12} and \textsf{E12} and calculates the totals for these columns.
	\item Click cell \textsf{F11}.
	\item Click the Copy button in the Home tab of the Ribbon.
	\item Click cell \textsf{F12}, then click the Paste button in the Home tab of the Ribbon. Since we now have totals in row 12, we can paste the percent change formula into this row.
\end{enumerate}

Figure \ref{02:fig12} shows the output of the SUM function that was added to cells \textsf{C12}, \textsf{D12}, and \textsf{E12}. In addition, the percent change formula was copied and pasted into cell \textsf{F12}. Notice that this version of the budget is planning a 1.7\% decrease in spending compared to last year.

\begin{figure}[H]
	\centering
	\includegraphics[width=\maxwidth{.95\linewidth}]{gfx/ch02_fig12}
	\caption{Results of the SUM Function in the Budget Detail Worksheet}
	\label{02:fig12}
\end{figure}

\begin{center}
	\begin{infobox}{Integrity Check}
		\textbf{Cell Ranges in Statistical Functions}
		\\
		\\
		When you intend to use a statistical function on a range of cells in a worksheet, make sure there are two cell locations separated by a colon and not a comma. If you enter two cell locations separated by a comma, the function will produce an output but it will be applied to only two cell locations instead of a range of cells. For example, the SUM function shown in Figure 2.13 will add only the values in cells \textsf{C3} and \textsf{C11}, not the range \textsf{C3:C11}.
	\end{infobox}
\end{center}

\begin{figure}[H]
	\centering
	\includegraphics[width=\maxwidth{.95\linewidth}]{gfx/ch02_fig13}
	\caption{SUM Function Adding Two Cell Locations}
	\label{02:fig13}
\end{figure}

\subsection{Absolute References (Calculating Percent of Totals)}

\textit{Data file: Continue with CH2 Personal Budget.}

Since totals were added to row 12 of the \textbf{Budget Detail} worksheet, a percent of total calculation can be added to Column B beginning in cell \textsf{B3}. The percent of total calculation shows the percentage for each value in the Annual Spend column with respect to the total in cell \textsf{D12}. However, after the formula is created, it will be necessary to turn off Excel's relative referencing feature before copying and pasting the formula to the rest of the cell locations in the column. Turning off Excel's relative referencing feature is accomplished through an absolute reference. The following steps explain how this is done:

\begin{enumerate}
	\item Click cell \textsf{B3} in the \textbf{Budget Detail} worksheet.
	\item Type an equal sign $ = $.
	\item Click cell \textsf{D3}.
	\item Type a forward slash $ / $.
	\item Click cell \textsf{D12}.
	\item Press the \keystroke{Enter} key. You will see that Household Utilities represents $ 16.7\% $ of the Annual Spend budget (see Figure \ref{02:fig14}).
\end{enumerate}

\begin{figure}[H]
	\centering
	\includegraphics[width=\maxwidth{.95\linewidth}]{gfx/ch02_fig14}
	\caption{Adding a Formula to Calculate the Percent of Total}
	\label{02:fig14}
\end{figure}

Figure \ref{02:fig14} shows the completed formula that is calculating the percentage that Household Utilities Annual Spend represents to the total Annual Spend for the budget (see cell \textsf{B3}). Normally, we would copy this formula and paste it into the range \textsf{B4:B11}. However, because of relative referencing, both cell references will increase by one row as the formula is pasted into the cells below \textsf{B3}. This is fine for the first cell reference in the formula (\textsf{D3}) but not for the second cell reference (\textsf{D12}). Figure \ref{02:fig15} illustrates what happens if we paste the formula into the range \textsf{B4:B12} in its current state. Notice that Excel produces the $ \#DIV/0 $ error code. This means that Excel is trying to divide a number by zero, which is impossible. Looking at the formula in cell \textsf{B4}, you see that the first cell reference was changed from \textsf{D3} to \textsf{D4}. This is fine because we now want to divide the Annual Spend for Insurance by the total Annual Spend in cell \textsf{D12}. However, Excel has also changed the \textsf{D12} cell reference to \textsf{D13}. Because cell location \textsf{D13} is blank, the formula produces the $ \#DIV/0 $ error code.

\begin{figure}[H]
	\centering
	\includegraphics[width=\maxwidth{.95\linewidth}]{gfx/ch02_fig15}
	\caption{$ \#DIV/0 $ Error from Relative Referencing}
	\label{02:fig15}
\end{figure}

To eliminate the divide-by-zero error shown in Figure \ref{02:fig15} we must add an absolute reference to cell \textsf{D12} in the formula. An absolute reference prevents relative referencing from changing a cell reference in a formula. This is also referred to as locking a cell. The following explains how this is accomplished.

\begin{enumerate}
	\item Double click cell \textsf{B3}.
	\item Place the mouse pointer in front of \textsf{D12} and click. The blinking cursor should be in front of the D in the cell reference \textsf{D12}.
	\item Press the \keystroke{F4} key. You will see a dollar sign ($ \$ $) added in front of the column letter D and the row number 12. You can also type the dollar signs in front of the column letter and row number.
	\item Press the \keystroke{Enter} key.
	\item Click cell \textsf{B3}.
	\item Click the Copy button in the Home tab of the Ribbon.
	\item Highlight the range \textsf{B4:B11}.
	\item Click the Paste button in the Home tab of the Ribbon.
\end{enumerate}

Figure \ref{02:fig16} shows the percent of total formula with an absolute reference added to \textsf{D12}. Notice that in cell \textsf{B4}, the cell reference remains \textsf{D12} instead of changing to \textsf{D13} as shown in Figure \ref{02:fig15}. Also, you will see that the percentages are being calculated in the rest of the cells in the column, and the divide-by-zero error is now eliminated.

\begin{figure}[H]
	\centering
	\includegraphics[width=\maxwidth{.95\linewidth}]{gfx/ch02_fig16}
	\caption{Adding an Absolute Reference to a Cell Reference in a Formula}
	\label{02:fig16}
\end{figure}

\begin{center}
	\begin{sklbox}{Skill Refresher}
		\textbf{Absolute References}
		\\
		\begin{itemize}
			\setlength{\itemsep}{0pt}
			\setlength{\parskip}{0pt}
			\setlength{\parsep}{0pt}
			
			\item Click in front of the column letter of a cell reference in a formula or function that you do not want altered when the formula or function is pasted into a new cell location.
			\item Press the \keystroke{F4} key or type a dollar sign ($ \$ $) in front of the column letter and row number of the cell reference.

		\end{itemize}
	\end{sklbox}
\end{center}

\subsection{The Count Function}

\textit{Data file: Continue with CH2 Personal Budget.}

The next function that we will add to the \textbf{Budget Detail} worksheet is the COUNT function. The COUNT function is used to determine how many cells in a range contain a numeric entry. The COUNT function will not work for counting text or other non-numeric entries. For the \textbf{Budget Detail} worksheet, we will use the COUNT function to count the number of items that are planned in the Annual Spend column (Column D). The following explains how the COUNT function is added to the worksheet by using the function list.

\begin{enumerate}
	\item Click cell \textsf{D13} in the \textbf{Budget Detail} worksheet.
	\item Type an equal sign $ = $.
	\item Type the letter C.
	\item Click the down arrow on the scroll bar of the function list (see Figure \ref{02:fig17}) and find the word COUNT.
	\item Double click the word COUNT from the function list.
	\item Highlight the range \textsf{D3:D11}.
	\item You can type a closing parenthesis $ ) $ and then press the \keystroke{Enter} key, or simply press the \keystroke{Enter} key and Excel will close the function for you. The function produces an output of 9 since there are 9 items planned on the worksheet.
\end{enumerate}

Figure \ref{02:fig17} shows the function list box that appears after completing steps 2 and 3 for the COUNT function. The function list provides an alternative method for adding a function to a worksheet.

\begin{figure}[H]
	\centering
	\includegraphics[width=\maxwidth{.95\linewidth}]{gfx/ch02_fig17}
	\caption{Using the Function List to Add the COUNT Function}
	\label{02:fig17}
\end{figure}

Figure \ref{02:fig18} shows the output of the COUNT function after pressing the \keystroke{Enter} key. The function counts the number of cells in the range \textsf{D3:D11} that contain a numeric value. The result of 9 indicates that there are 9 categories planned for this budget.

\begin{figure}[H]
	\centering
	\includegraphics[width=\maxwidth{.95\linewidth}]{gfx/ch02_fig18}
	\caption{Completed COUNT Function in the Budget Detail Worksheet}
	\label{02:fig18}
\end{figure}

\subsection{The Average Function}

The next function we will add to the \textbf{Budget Detail} worksheet is the AVERAGE function. This function is used to calculate the arithmetic mean for a group of numbers. For the \textbf{Budget Detail} worksheet, we will use the function to calculate the average of the values in the Annual Spend column. We will add this to the worksheet by using the Function Library. The following steps explain how this is accomplished.

\begin{enumerate}
	\item Click cell \textsf{D14} in the \textbf{Budget Detail} worksheet.
	\item Click the Formulas tab on the Ribbon.
	\item Click the More Functions button in the Function Library group of commands.
	\item Place the mouse pointer over the Statistical option from the drop-down list of options.
	\item Click the AVERAGE function name from the list of functions that appear in the menu (see Figure \ref{02:fig19}). This opens the Function Arguments dialog box.
	\item Click the Collapse Dialog button in the Function Arguments dialog box (see Figure \ref{02:fig20}).
	\item Highlight the range \textsf{D3:D11}.
	\item Click the Expand Dialog button in the Function Arguments dialog box (see Figure \ref{02:fig21}). You can also press the \keystroke{Enter} key to get the same result.
	\item Click the OK button on the Function Arguments dialog box. This adds the AVERAGE function to the worksheet.
\end{enumerate}

Figure \ref{02:fig19} illustrates how a function is selected from the Function Library in the Formulas tab of the Ribbon.

\begin{figure}[H]
	\centering
	\includegraphics[width=\maxwidth{.95\linewidth}]{gfx/ch02_fig19}
	\caption{Selecting the AVERAGE Function from the Function Library}
	\label{02:fig19}
\end{figure}

Figure \ref{02:fig20} shows the Function Arguments dialog box. This appears after a function is selected from the Function Library. The Collapse Dialog button is used to hide the dialog box so a range of cells can be highlighted on the worksheet and then added to the function.

\begin{figure}[H]
	\centering
	\includegraphics[width=\maxwidth{.95\linewidth}]{gfx/ch02_fig20}
	\caption{Function Arguments Dialog Box}
	\label{02:fig20}
\end{figure}

Figure \ref{02:fig21} shows how a range of cells can be selected from the Function Arguments dialog box once it has been collapsed.

\begin{figure}[H]
	\centering
	\includegraphics[width=\maxwidth{.95\linewidth}]{gfx/ch02_fig21}
	\caption{Selecting a Range from the Function Arguments Dialog Box}
	\label{02:fig21}
\end{figure}

Figure \ref{02:fig22} shows the Function Arguments dialog box after the cell range is defined for the AVERAGE function. The dialog box shows the result of the function before it is added to the cell location. This allows you to assess the function output to determine whether it makes sense before adding it to the worksheet.

\begin{figure}[H]
	\centering
	\includegraphics[width=\maxwidth{.95\linewidth}]{gfx/ch02_fig22}
	\caption{Function Arguments Dialog Box after a Cell Range Is Defined for a Function}
	\label{02:fig22}
\end{figure}

Figure \ref{02:fig23} shows the completed AVERAGE function in the Budget Detail worksheet. The output of the function shows that on average we expect to spend $ \$1,994 $ for each of the categories listed in Column A of the budget. This average spend calculation per category can be used as an indicator to determine which categories are costing more or less than the average budgeted spend dollars.

\begin{figure}[H]
	\centering
	\includegraphics[width=\maxwidth{.95\linewidth}]{gfx/ch02_fig23}
	\caption{Completed AVERAGE Function}
	\label{02:fig23}
\end{figure}

\subsection{The Max and Min Functions}

\textit{Data file: Continue with CH2 Personal Budget.}

The final two statistical functions that we will add to the Budget Detail worksheet are the MAX and MIN functions. These functions identify the highest and lowest values in a range of cells. The following steps explain how to add these functions to the Budget Detail worksheet.

\begin{enumerate}
	\item Click cell \textsf{D15} in the Budget Detail worksheet.
	\item Type an equal sign $ = $.
	\item Type the word MIN.
	\item Type an open parenthesis $ ( $.
	\item Highlight the range \textsf{D3:D11}.
	\item Type a closing parenthesis $ ) $ and press the \keystroke{Enter} key, or simply press the \keystroke{Enter} key and Excel will close the function for you. The MIN function produces an output of $ \$1,200 $, which is the lowest value in the Annual Spend column (see Figure \ref{02:fig24}).
	\item Click cell \textsf{D16}.
	\item Type an equal sign $ = $.
	\item Type the word MAX.
	\item Type an open parenthesis $ ( $.
	\item Highlight the range \textsf{D3:D11}.
	\item Type a closing parenthesis $ ) $ and press the \keystroke{Enter} key, or simply press the \keystroke{Enter} key and Excel will close the function for you. The MAX function produces an output of $ \$3,500 $. This is the highest value in the Annual Spend column (see Figure \ref{02:fig25}).
\end{enumerate}

\begin{figure}[H]
	\centering
	\includegraphics[width=\maxwidth{.95\linewidth}]{gfx/ch02_fig24}
	\caption{MIN Function Added to the Budget Detail Worksheet}
	\label{02:fig24}
\end{figure}

\begin{figure}[H]
	\centering
	\includegraphics[width=\maxwidth{.95\linewidth}]{gfx/ch02_fig25}
	\caption{MAX Function Added to the Budget Detail Worksheet}
	\label{02:fig25}
\end{figure}

\begin{center}
	\begin{sklbox}{Skill Refresher}
		\textbf{Statistical Functions}
		\\
		\begin{itemize}
			\setlength{\itemsep}{0pt}
			\setlength{\parskip}{0pt}
			\setlength{\parsep}{0pt}

			\item Type an equal sign $ = $.
			\item Type the function name followed by an open parenthesis $ ( $ or double click the function name from the function list.
			\item Highlight a range on a worksheet or click individual cell locations followed by commas.
			\item Type a closing parenthesis $ ) $ and press the \keystroke{Enter} key or press the \keystroke{Enter} key to close the function.
			
		\end{itemize}
	\end{sklbox}
\end{center}

\section{Copy and Paste Formulas (Pasting Without Formats)}

\textit{Data file: Continue with CH2 Personal Budget.}

As shown in Figure \ref{02:fig25}, the COUNT, AVERAGE, MIN, and MAX functions are summarizing the data in the Annual Spend column. You will also notice that there is space to copy and paste these functions under the LY Spend column. This allows us to compare what we spent last year and what we are planning to spend this year. Normally, we would simply copy and paste these functions into the range \textsf{E13:E16}. However, you may have noticed the double-line style border that was used around the perimeter of the range \textsf{B13:E16}. If we used the regular Paste command, the double line on the right side of the range \textsf{E13:E16} would be replaced with a single line. Therefore, we are going to use one of the Paste Special commands to paste only the functions without any of the formatting treatments. This is accomplished through the following steps.

\begin{enumerate}
	\item Highlight the range D13:D16 in the Budget Detail worksheet.
\item Click the Copy button in the Home tab of the Ribbon.
\item Click cell E13.
\item Click the down arrow below the Paste button in the Home tab of the Ribbon.
\item Click the Formulas option from the drop-down list of buttons (see Figure \ref{02:fig26}).
\end{enumerate}

Figure \ref{02:fig26} shows the list of buttons that appear when you click the down arrow below the Paste button in the Home tab of the Ribbon. One thing to note about these options is that you can preview them before you make a selection by dragging the mouse pointer over the options. As shown in the figure, when the mouse pointer is placed over the Formulas button, you can see how the functions will appear before making a selection. Notice that the double-line border does not change when this option is previewed. That is why this selection is made instead of the regular Paste option.

\begin{figure}[H]
	\centering
	\includegraphics[width=\maxwidth{.95\linewidth}]{gfx/ch02_fig26}
	\caption{Paste Formulas Option}
	\label{02:fig26}
\end{figure}

\begin{center}
	\begin{sklbox}{Skill Refresher}
		\textbf{Paste Formulas}
		\\
		\begin{itemize}
			\setlength{\itemsep}{0pt}
			\setlength{\parskip}{0pt}
			\setlength{\parsep}{0pt}
			
			\item Click a cell location containing a formula or function.
			\item Click the Copy button in the Home tab of the Ribbon.
			\item Click the cell location or cell range where the formula or function will be pasted.
			\item Click the down arrow below the Paste button in the Home tab of the Ribbon.
			\item Click the Formulas button under the Paste group of buttons.
			
		\end{itemize}
	\end{sklbox}
\end{center}

\section{Sorting Data (Multiple Levels)}

\textit{Data file: Continue with CH2 Personal Budget.}

The \textbf{Budget Detail} worksheet shown in Figure \ref{02:fig26} is now producing several mathematical outputs through formulas and functions. The outputs allow you to analyze the details and identify trends as to how money is being budgeted and spent. Before we draw some conclusions from this worksheet, we will sort the data based on the Percent of Total column. Sorting is a powerful tool that enables you to analyze key trends in any data set. Sorting will be covered thoroughly in a later chapter, but will be briefly introduced here. For the purposes of the \textbf{Budget Detail} worksheet, we want to set multiple levels for the sort order. This is accomplished through the following steps:

\begin{enumerate}
	\item Highlight the range \textsf{A2:F11} in the \textbf{Budget Detail} worksheet.
	\item Click the Data tab in the Ribbon.
	\item Click the Sort button in the Sort \& Filter group of commands. This opens the Sort dialog box, as shown in Figure \ref{02:fig27}.
	\item Click the down arrow next to the ``Sort by'' box.
	\item Click the Percent of Total option from the drop-down list.
	\item Click the down arrow next to the sort Order box.
	\item Click the Largest to Smallest option.
	\item Click the Add Level button. This allows you to set a second level for any duplicate values in the Percent of Total column.
	\item Click the down arrow next to the ``Then by'' box.
	\item Select the \textit{LY Spend} option. Leave the Sort Order as Smallest to Largest
	\item Click the OK button at the bottom of the Sort dialog box.
	\item Save the \textbf{Ch2 Personal Budget} file.
\end{enumerate}

\begin{figure}[H]
	\centering
	\includegraphics[width=\maxwidth{.95\linewidth}]{gfx/ch02_fig27}
	\caption{Sort Dialog Box}
	\label{02:fig27}
\end{figure}

Figure \ref{02:fig28} shows the Budget Detail worksheet after it has been sorted. Notice that there are three identical values in the Percent of Total column. This is why a second sort level had to be created for this worksheet. The second sort level arranges the values of $ 8.4\% $ based on the values in the LY Spend column in ascending order. Excel gives you the option to set as many sort levels as necessary for the data contained in a worksheet.

\begin{figure}[H]
	\centering
	\includegraphics[width=\maxwidth{.95\linewidth}]{gfx/ch02_fig28}
	\caption{Budget Detail Worksheet after Sorting}
	\label{02:fig28}
\end{figure}

\begin{center}
	\begin{sklbox}{Skill Refresher}
		\textbf{Sorting Data (Multiple Levels)}
		\\
		\begin{itemize}
			\setlength{\itemsep}{0pt}
			\setlength{\parskip}{0pt}
			\setlength{\parsep}{0pt}
			
			\item Highlight a range of cells to be sorted.
			\item Click the Data tab of the Ribbon.
			\item Click the Sort button in the Sort \& Filter group.
			\item Select a column from the ``Sort by'' drop-down list in the Sort dialog box.
			\item Select a sort order from the Order drop-down list in the Sort dialog box.
			\item Click the Add Level button in the Sort dialog box.
			\item Repeat Steps 4 and 5.
			\item Click the OK button on the Sort dialog box.
			
		\end{itemize}
	\end{sklbox}
\end{center}

Now that the \textbf{Budget Detail} worksheet is sorted, a few key trends can be easily identified. The worksheet clearly shows that the top three categories as a percentage of total budgeted spending for the year are Taxes, Household Utilities, and Food. All three categories are necessities (or realities) of life and typically require a significant amount of income for most households. Looking at the Percent Change column, we can see how our planned spending is expected to change from last year. This is perhaps the most important column on the worksheet because it allows you to assess whether your plan is realistic. You will see that there are no changes planned for Taxes and Household Utilities. While Taxes can change from year to year, it is not too difficult to predict what they will be. In this case, we are assuming that there are no changes to the tax costs for our budget. We are also planning no change in Household Utilities. These costs can fluctuate from year to year as well. However, you can take measures to reduce costs, such as using less electricity, turning off heat when no one is in the house, keeping track of your wireless minutes so you do not go over the maximum allowed in your plan, and so on. As a result, there is no change in planned spending for Household Utilities because we will assume that any rate increases will be offset with a decrease in usage. The third item that is planned not to change is Insurance. Insurance policies for cars and homes can change, but as is true for taxes, the changes are predictable. Therefore, we are assuming no changes in our insurance policy.

The first big change that is noticeable in the worksheet is the Food and Entertainment categories in rows 5 and 6 (see definitions in Table \ref{02:tab01}). The Percent Change column indicates that there is an $ 11.1\% $ decrease in Entertainment spending and an $ 11.1\% $ increase in Food spending. This is logical because if you plan to eat in restaurants less frequently, you will be eating at home more frequently. Although this makes sense in theory, it may be hard to do in practice. Dinners and parties with friends may be tough to turn down. However, the entire process of maintaining a budget is based on discipline, and it certainly takes a significant amount of discipline to plan targets for yourself and stick to them.

A few other points to note are the changes in the Gasoline and Vacation categories. If you commute to school or work, the price of gas can have a significant impact on your budget. It is important to be realistic if gas prices are increasing, and you should reflect these increases in your budget. To compensate for the increased spending for gas, the spending plan for vacations has been reduced by $ 25\% $. Budgeting often requires a certain degree of creativity. Although the Vacation budget has been reduced, there is still money you can set aside to make plans for spring break or winter break.

Finally, the budget shows a decrease in Miscellaneous spending of $ 19.8\% $. This was defined as a group containing several expenses, such as textbooks, school supplies, software updates, and so on (see Table \ref{02:tab01}). You may be able to reduce your spending in this category if you can use items such as online textbooks. This reduction in spending can free up funds for Clothes, a spend category that has increased by $ 20\% $. We will continue to develop the Personal Budget workbook further in Section \ref{ch02:functions_personal}: \nameref{ch02:functions_personal}.

\begin{center}
	\begin{tkwbox}{Key Take-Aways}
		\textbf{Save}
		\\
		\begin{itemize}
			\setlength{\itemsep}{0pt}
			\setlength{\parskip}{0pt}
			\setlength{\parsep}{0pt}
			
			\item Statistical functions are used when a mathematical process is required for a range of cells, such as summing the values in several cell locations. For these computations, functions are preferable to formulas because adding many cell locations one at a time to a formula can be very time-consuming.
			\item Statistical functions can be created using cell ranges or selected cell locations separated by commas. Make sure you use a cell range (two cell locations separated by a colon) when applying a statistical function to a
			contiguous range of cells.
			\item To prevent Excel from changing the cell references in a formula or function when they are pasted to a new cell location, you must use an absolute reference. You can do this by placing a dollar sign (\$) in front of the column letter and row number of a cell reference.
			\item The $ \#DIV/0 $ error appears if you create a formula that attempts to divide a constant or the value in a cell reference by zero.
			\item The Paste Formulas option is used when you need to paste formulas without any formatting treatments into cell locations that have already been formatted.
			\item You need to set multiple levels, or columns, in the Sort dialog box when sorting data that contains several duplicate values.
			
		\end{itemize}
	\end{tkwbox}
\end{center}

\section{Functions for Personal Finance}\label{ch02:functions_personal}

\begin{center}
	\begin{objbox}{Learning Objectives}
		\begin{itemize}
			\setlength{\itemsep}{0pt}
			\setlength{\parskip}{0pt}
			\setlength{\parsep}{0pt}
			
			\item Understand the fundamentals of loans and leases.
			\item Use the PMT function to calculate monthly mortgage payments on a house.
			\item Use the PMT function to calculate monthly lease payments for an automobile.
			\item Learn how to summarize data in a workbook by using worksheet links to create a summary worksheet.

		\end{itemize}
	\end{objbox}
\end{center}

In this section, we continue to develop the \textbf{Personal Budget} workbook. Notable items that are missing from the Budget Detail worksheet are the payments you might make for a car or a home. This section demonstrates Excel functions used to calculate lease payments for a car and to calculate mortgage payments for a house.

\subsection{The Fundamentals of Loans and Leases}

One of the functions we will add to the \textbf{Personal Budget} workbook is the PMT function. This function calculates the payments required for a loan or a lease. However, before demonstrating this function, it is important to cover a few fundamental concepts on loans and leases.

A loan is a contractual agreement in which money is borrowed from a lender and paid back over a specific period of time. The amount of money that is borrowed from the lender is called the principal of the loan. The borrower is usually required to pay the principal of the loan plus interest. When you borrow money to buy a house, the loan is referred to as a mortgage. This is because the house being purchased also serves as collateral to ensure payment. In other words, the bank can take possession of your house if you fail to make loan payments. As shown in Table \ref{02:tab05}, there are several key terms related to loans and leases.

{\small
	\begin{longtable}{p{0.75in}p{3.5in}}
		\textbf{Term} & \textbf{Definition}\endhead
		\hline \\
		Collateral & Any item of value that is used to secure a loan to ensure payments to the lender.\\
		Down Payment & The amount of cash paid toward the purchase of a house. If you are paying $ 20\% $ down, you are paying $ 20\% $ of the cost of the house in cash and are borrowing the rest from a lender.\\
		Interest Rate & The interest that is charged to the borrower as a cost for borrowing money.\\
		Mortgage & A loan where property is put up for collateral.\\
		Principle & The amount of money that has been borrowed.\\
		Residual Value & The estimated selling price of a vehicle at a future point in time.\\
		Terms & The amount of time you have to repay a loan.\\
		\caption{Key Terms for Loans and Leases}
		\label{02:tab05}
	\end{longtable}
}

Figure \ref{02:fig29} shows an example of an amortization table for a loan. A lender is required by law to provide borrowers with an amortization table when a loan contract is offered. The table in the figure shows how the payments of a loan would work if you borrowed $ \$100,000 $ from a lender and agreed to pay it back over $ 10 $ years at an interest rate of $ 5\% $. You will notice that each time you make a payment, you are paying the bank an interest fee plus some of the loan principal. Each year the amount of interest paid to the bank decreases and the amount of money used to pay off the principal increases. This is because the bank is charging you interest on the amount of principal that has not been paid. As you pay off the principal, the interest rate is applied to a lower number, which reduces your interest charges. Finally, the figure shows that the sum of the values in the Interest Payment column is $ \$29,505 $. This is how much it costs you to borrow this money over $ 10 $ years. Indeed, borrowing money is not free. It is important to note that to simplify this example, the payments were calculated on an annual basis. However, most loan payments are made on a monthly basis.

\begin{figure}[H]
	\centering
	\includegraphics[width=\maxwidth{.95\linewidth}]{gfx/ch02_fig29}
	\caption{Example of an Amortization Table}
	\label{02:fig29}
\end{figure}

A lease is a contract in which you, the lessee, use an asset such as a car or a piece of equipment and you agree to make regular payments to the owner or the lessor. When you lease a car, the manufacturer or a leasing company retains ownership of the vehicle and you agree to make regular payments for a specific period of time. The amount of money you pay depends on the price of the car, the terms of the lease contract, and the car's expected residual value at the end of the lease. The calculation of lease payments is similar to the calculation of loan payments. However, when you lease a car, you pay only the value of the car that is used. For example, suppose you are leasing a car that is priced at $ \$25,000 $ for $ 4 $ years at an interest rate of $ 5\% $. The residual value of the car is $ \$10,000 $. This means the car will lose $ \$15,000 $ of its value over $ 4 $ years. Another way to state this is that the car will depreciate $ \$15,000 $. A lease will be structured so that you pay this $ \$15,000 $ in depreciation. However, the interest charges will be based on the purchase price of $ \$25,000 $. We will look at a demonstration of leasing a car as well as buying a home in the next section.

\subsection{The Pmt (Payment) Function for Loans}

\textit{Data file: Continue with CH2 Personal Budget.}

If you own a home, your mortgage payments are a major component of your household budget. If you are planning to buy a home, having a clear understanding of your monthly payments is critical for maintaining strong financial health. In Excel, mortgage payments are conveniently calculated through the PMT (payment) function. This function is more complex than the statistical functions covered in Section \ref{ch02:statistical_functions}: \nameref{ch02:statistical_functions}. With statistical functions, you are required to add only a range of cells or selected cells within the parentheses of the function, also known as the argument. With the PMT function, you must accurately define a series of arguments in order for the function to produce a reliable output. Table 2.6 lists the arguments for the PMT function. It is helpful to review the key loan and lease terms in Table \ref{02:tab05} before reviewing the PMT function arguments.

{\small
	\begin{longtable}{p{0.75in}p{3.5in}}
		\textbf{Argument} & \textbf{Definition}\endhead
		\hline \\
		Rate & This is the interest rate the lender is charging the borrower. The interest rate is usually quoted in annual terms, so you have to divide this rate by $ 12 $ if you are calculating monthly payments.\\
		Nper & The argument letters stand for number of periods. This is the term of the loan, which is the amount of time you have to repay the bank. This is usually quoted in years, so you have to multiply the years by $ 12 $ if you are calculating monthly payments.\\
		Pv & The argument letters stand for present value. This is the principal of the loan or the amount of money that is borrowed. When defining this argument, a minus sign must precede the cell location or value. For leases, this argument is used for the price of the item being leased.\\
		{[Fv]} & The argument letters stand for future value. The brackets around the argument indicate that it is not always necessary to define it. It is used if there is a lump-sum payment that will be made at the end of the loan terms. This is also used for the residual value of a lease. If it is not defined, Excel will assume that it is zero.\\
		{[Type]} & This argument can be defined with either a $ 1 $ or a $ 0 $. The number $ 1 $ is used if payments are made at the beginning of each period. A $ 0 $ is used if payments are made at the end of each period. The argument is in brackets because it does not have to be defined if payments are made at the end of each period. Excel assumes that this argument is $ 0 $ if it is not defined.\\
		\caption{Arguments for the PMT Function}
		\label{02:tab06}
	\end{longtable}
}

By default, the result of the PMT function in Excel is shown as a negative number. This is because it represents an outgoing payment. When making a mortgage or car payment, you are paying money out of your pocket or bank account. Depending on the type of work that you do, your employer may want you to leave your payments negative or they may ask you to format them as positive numbers. In the following assignments, the payments calculated using the PMT function will be made positive to make them easier to work with. To do this, when defining the PV argument (amount of money borrowed) in the PMT dialog box, a minus sign must precede the cell location or value (see the PV argument in Figure \ref{02:fig32}).

We will use the PMT function in the \textbf{Personal Budget} workbook to calculate the monthly mortgage payments for a house. These calculations will be made in the Mortgage Payments worksheet and then displayed in the Budget Summary worksheet through a cell reference link. So far we have demonstrated several methods for adding functions to a worksheet. The following steps explain a new method using the Insert Function command for adding the PMT function.

\begin{enumerate}
	\item Click the \textbf{Mortgage Payments} worksheet tab.
	\item Click cell \textsf{B5}.
	\item Click the Formulas tab on the Ribbon.
	\item Click the Insert Function button (see Figure \ref{02:fig30}). This opens the Insert Function dialog box, which can be used for searching all functions in Excel.
	\item In the ``Search for a function:'' input box at the top of the Insert Function dialog box, type mortgage payments (see Figure \ref{02:fig31}). Note that the current description in the ``Search for a function:'' input box will already be highlighted. You can begin typing and the description will be replaced with your entry.
	\item Click the Go button in the upper right side of the Insert Function dialog box. This adds all the Excel functions that match your description in the ``Select a function:'' box in the lower half of the Insert Function dialog box (see Figure \ref{02:fig31}).
	\item Click the PMT option in the ``Select a function:'' box in the lower half of the Insert Function dialog box.
	\item Click the OK button at the lower right side of the Insert Function dialog box. This will open the Function Arguments dialog box.
\end{enumerate}

\begin{figure}[H]
	\centering
	\includegraphics[width=\maxwidth{.95\linewidth}]{gfx/ch02_fig30}
	\caption{Mortgage Payments Worksheet}
	\label{02:fig30}
\end{figure}

\begin{figure}[H]
	\centering
	\includegraphics[width=\maxwidth{.95\linewidth}]{gfx/ch02_fig31}
	\caption{Insert Function Dialog Box}
	\label{02:fig31}
\end{figure}

\begin{enumerate}[resume]
	\item Click the Collapse Dialog button next to the Rate argument in the Function Arguments dialog box. This will be the first argument defined for the function (see Figure \ref{02:fig32})
	\item Click cell \textsf{B3} on the worksheet. This is the rate being charged on the loan.
	\item Type a forward slash $ / $ for division.
	\item Type the number $ 12 $. Since our goal is to calculate the monthly payments for the loan, we need to divide the rate, which is stated in annual terms, by $ 12 $. This converts the annual rate to a monthly rate.
	\item Press the \keystroke{Enter} key on your keyboard. This returns the Function Arguments dialog box to its expanded form. You will also see that the Rate argument is now defined.
	\item Click the Collapse Dialog button next to the Nper argument in the Function Arguments dialog box. This is the second argument we define in the function.
	\item Click cell \textsf{B4} on the worksheet. This is the term or the amount of time we have to repay the loan.
	\item Type an asterisk $ * $ for multiplication.
	\item Type the number $ 12 $. Since our goal is to calculate the monthly payments for the loan, we need to multiply the terms of the loan by $ 12 $. This converts the terms of the loan from years to months.
	\item Press the \keystroke{Enter} key on your keyboard. This returns the Function Arguments dialog box to its expanded form. You will also see that the Nper argument is now defined.
	\item Click the Collapse Dialog button next to the Pv argument in the Function Arguments dialog box. This is the third argument we will define in the function.
	\item Type a minus sign $ - $. When defining the Pv argument of the PMT function, any cell location or value must be preceded with a minus sign.
	\item Click cell \textsf{B2} on the worksheet. This is the principal of the loan.
	\item Press the \keystroke{Enter} key on your keyboard. You will now see the Rate, Nper, and Pv arguments defined for the function.
	\item Click the OK button at the bottom of the Function Arguments dialog box. The function will now be placed into the worksheet. Since we are not paying any lump sums of money at the end of the loan, there is no need to define the Fv argument. Also, we will assume that the monthly mortgage payments will be made at the end of each month. Therefore, there is no need to define the Type argument.
\end{enumerate}

\begin{center}
	\begin{shtcutbox}{Keyboard Shortcuts}
		\textbf{Functions}
		\\
		\begin{itemize}
			\setlength{\itemsep}{0pt}
			\setlength{\parskip}{0pt}
			\setlength{\parsep}{0pt}
			
			\item \textbf{Insert Function}: Hold the \keystroke{Shift} key while pressing the \keystroke{F3} key.
			\item \textbf{Function Arguments}: After the equal sign $ = $ and function name are typed into cell a location, hold down the \keystroke{Ctrl} key and press the letter A on your keyboard.		
		\end{itemize}
	
	\end{shtcutbox}
\end{center}

Figure \ref{02:fig32} shows the completed Function Arguments dialog box for the PMT function. Notice that the dialog box shows the values for the Rate and Nper arguments. The Rate is divided by $ 12 $ to convert the annual interest rate to a monthly interest rate. The Nper argument is multiplied by $ 12 $ to convert the terms of the loan from years to months. Finally, the dialog box provides you with a definition for each argument. The definition appears when you click in the input box for the argument.

\begin{figure}[H]
	\centering
	\includegraphics[width=\maxwidth{.95\linewidth}]{gfx/ch02_fig32}
	\caption{Function Arguments Dialog Box for the PMT Function}
	\label{02:fig32}
\end{figure}

Figure \ref{02:fig33} shows the final appearance of the \textbf{Mortgage Payments} worksheet after the PMT function is added. The result of the function in cell \textsf{B5} will be displayed in the Budget Summary worksheet.

\begin{figure}[H]
	\centering
	\includegraphics[width=\maxwidth{.95\linewidth}]{gfx/ch02_fig33}
	\caption{Mortgage Payments Worksheet with the PMT Function}
	\label{02:fig33}
\end{figure}

\begin{center}
	\begin{infobox}{Integrity Check}
		\textbf{Comparable Arguments for PMT Function}
		\\
		\\
		When using functions such as PMT, make sure the arguments are defined in comparable terms. For example, if you are calculating the monthly payments of a loan, make sure both the Rate and Nper argument are expressed in terms of months. The function will produce an erroneous result if one argument is expressed in years while the other is expressed in months.		
	\end{infobox}
\end{center}

\subsection{The Pmt (Payment) Function for Leases}

In addition to calculating the mortgage payments for a home, the PMT function will be used in the Personal Budget workbook to calculate the lease payments for a car. The details for the lease payments are found in the Car Lease Payments worksheet. Similar to the statistical functions, we can either type the PMT function directly into a cell or use the Insert Function button. However, you must know the definitions for each argument of the function and understand how these arguments need to be defined based on your objective. The terms for loans and leases are in Table \ref{02:tab05} and the definitions for the arguments of the PMT function are in Table \ref{02:tab06}. The following steps explain how the PMT function is added to the \textbf{Personal Budget} workbook to calculate the lease payments for a car.

\begin{enumerate}
	\item Click cell \textsf{B6} in the \textbf{Car Lease Payments} worksheet.
	\item Click the Formulas tab on the Ribbon.
	\item Click the Financial button in the Function Library group. This opens the Financial Function drop down list.
	\item Scroll down and click on the PMT Function in the drop down list. This will open the PMT Function Arguments dialog box. See Figure \ref{02:fig34}.
	\item Click the Collapse Dialog button next to the Rate argument in the PMT Function Arguments dialog box. This will be the first argument defined for the monthly car lease payment.
	\item Click cell \textsf{B4}. This is the interest rate being charged for the lease.
	\item Type the forward slash $ / $ for division.
	\item Type the number $ 12 $. Since our goal is to calculate the monthly lease payments, we divide the interest rate by $ 12 $ to convert the annual rate to a monthly rate.
	\item Press the \keystroke{Enter} key on your keyboard. This returns the Function Arguments dialog box to its expanded form. You will also see that the Rate argument is now defined.
	\item Click the Collapse Dialog button next to the Nper argument in the Function Arguments dialog box. This is the second argument we define in the function.
	\item Click cell \textsf{B5}. This is the term or the length of time for the lease contract. Since the term is already expressed in months, we can just reference cell \textsf{B5} and move to the next argument. If the term was defined in years, instead of months, we would need to multiply the terms (years) of the loan by $ 12 $ to convert the years to months.
	\item Press the \keystroke{Enter} key on your keyboard. This returns the Function Arguments dialog box to its expanded form. You will also see that the Nper argument is now defined.
	\item Click the Collapse Dialog button next to the Pv argument in the Function Arguments dialog box. This is the third argument we will define in the function.
	\item Type a minus sign $ - $. Remember that cell locations or values used to define the Pv argument must be preceded with a minus sign.
	\item Click cell \textsf{B2} on the worksheet, which is the price of the car.
	\item Press the \keystroke{Enter} key on your keyboard. You will now see the Rate, Nper, and Pv arguments defined for the function.
	\item Click the Collapse Dialog button next to the Fv argument in the Function Arguments dialog box. This is the fourth argument we will define in the function.
	\item Click cell \textsf{B3} on the worksheet. This is the residual value of the car. Note that cell location and values used to define the [Fv] argument are NOT preceded by a minus sign.
	\item Press the \keystroke{Enter} key on your keyboard. You will now see the Rate, Nper, Pv and Fv arguments defined for the function.
	\item Click the Collapse Dialog button next to the Type argument in the Function Arguments dialog box. This is the fifth and last argument we will define in the function.
	\item Type the number $ 1 $. We will assume that the lease payments will be due at the beginning of each month. For payments made at the beginning of the period, a $ 1 $ will be entered in the Type argument box. For payments made at the end of the period, a $ 0 $ will be entered in the Type argument box.
	\item Press the \keystroke{Enter} key. You will now see the Rate, Nper, Pv, Fv and Type arguments defined for the function. See Figure \ref{02:fig34}.
	\item Click the OK button at the bottom of the Function Arguments dialog box. The function will now be placed into the worksheet.
\end{enumerate}

Figure \ref{02:fig34} shows how the the completed Function Arguments dialog box for the PMT function car lease should appear before pressing the OK button.

\begin{figure}[H]
	\centering
	\includegraphics[width=\maxwidth{.95\linewidth}]{gfx/ch02_fig34}
	\caption{Function Arguments Dialog Box for the PMT Lease Function}
	\label{02:fig34}
\end{figure}

Figure \ref{02:fig35} shows the result of the PMT function for the car lease. The monthly payments for this lease are $ \$206.56 $. This monthly payment will be displayed in the \textbf{Budget Summary} worksheet.

\begin{figure}[H]
	\centering
	\includegraphics[width=\maxwidth{.95\linewidth}]{gfx/ch02_fig35}
	\caption{Results of the PMT Function in the Car Lease Payments Worksheet}
	\label{02:fig35}
\end{figure}

\begin{center}
	\begin{sklbox}{Skill Refresher}
		\textbf{PMT Function}
		\\
		\begin{itemize}
			\setlength{\itemsep}{0pt}
			\setlength{\parskip}{0pt}
			\setlength{\parsep}{0pt}
			
			\item Type an equal sign $ = $.
			\item Type the letters PMT followed by an open parenthesis, or double click the function name from the function list.
			\item Define the Rate argument with a cell location that contains the rate being charged by the lender for the loan or lease. If the interest rate given is an annual rate, divide it by $ 12 $ to convert it to a monthly rate.
			\item Define the Nper argument with a cell location that contains the amount of time to repay the loan or lease. If the amount of time is in years, multiply it by $ 12 $ to convert it to number of months.
			\item Define the Pv argument with a cell location that contains the principal of the loan or the price of the item being leased. Cell locations or values used for this argument must be preceded by a minus sign.
			\item Define the [Fv] argument with a cell location that contains the residual value of the item being leased or the lump sum payment for a loan.
			\item Define the [Type] argument with a $ 1 $ if payments are made at the beginning of each period or $ 0 $ if payments are made at the end of each period.
			\item Type a closing parenthesis $ ) $.
			\item Press the \keystroke{Enter} key.
			
		\end{itemize}
	\end{sklbox}
\end{center}

\section{Linking Worksheets (Creating a Summary Worksheet)}

So far we have used cell references in formulas and functions, which allow Excel to produce new outputs when the values in the cell references are changed. Cell references can also be used to display values or the outputs of formulas and functions in cell locations on other worksheets. This is how data will be displayed on the \textbf{Budget Summary} worksheet in the \textbf{Personal Budget} workbook. Outputs from the formulas and functions that were entered into the \textbf{Budget Detail}, \textbf{Mortgage Payments}, and \textbf{Car Lease Payments} worksheets will be displayed on the \textbf{Budget Summary} worksheet through the use of cell references. The following steps explain how this is accomplished.

\begin{enumerate}
	\item Click cell \textsf{C3} in the \textbf{Budget Summary} worksheet.
	\item Type an equal sign $ = $.
	\item Click the \textbf{Budget Detail} worksheet tab.
	\item Click cell \textsf{D12} on the \textbf{Budget Detail} worksheet.
	\item Press the \keystroke{Enter} key on your keyboard. The output of the SUM function in cell \textsf{D12} on the \textbf{Budget Detail} worksheet will be displayed in cell \textsf{C3} on the \textbf{Budget Summary} worksheet.
\end{enumerate}

Figure \ref{02:fig36} shows how the cell reference appears in the \textbf{Budget Summary} worksheet. Notice that the cell reference \textsf{D12} is preceded by the \textbf{Budget Detail} worksheet name enclosed in apostrophes followed by an exclamation point ('Budget Detail'!) This indicates that the value displayed in the cell is referencing a cell location in the \textbf{Budget Detail} worksheet.

\begin{figure}[H]
	\centering
	\includegraphics[width=\maxwidth{.95\linewidth}]{gfx/ch02_fig36}
	\caption{Cell Reference Showing the Total Expenses in the Budget Summary Worksheet}
	\label{02:fig36}
\end{figure}

As shown in Figure \ref{02:fig36}, the \textbf{Budget Summary} worksheet is designed to show the expense budget for the mortgage payments and the auto lease payments. However, you will recall that we used the PMT function to calculate the monthly payments. In the \textbf{Budget Summary} worksheet, we need to show the total annual payments. As a result, we will create a formula that references cell locations in the \textbf{Mortgage Payments} and \textbf{Car Lease Payments} worksheets. The following steps explain how this is accomplished.

\begin{enumerate}
	\item Click cell \textsf{C4} in the \textbf{Budget Summary} worksheet.
	\item Type an equal sign $ = $.
	\item Click the \textbf{Mortgage Payments} worksheet tab.
	\item Click cell \textsf{B5} in the \textbf{Mortgage Payments} worksheet.
	\item Type an asterisk $ * $ for multiplication.
	\item Type the number $ 12 $. This multiplies the monthly payments by $ 12 $ to calculate the total payments required for the year. The formula in the formula bar should read: \formatfunc{='Mortgage Payments'!B5*12}
	\item Press the \keystroke{Enter} key on your keyboard. The value of multiplying the monthly mortgage payments by $ 12 $ is now displayed on the \textbf{Budget Summary} worksheet.
	\item Click cell \textsf{C5} on the \textbf{Budget Summary} worksheet.
	\item Type an equal sign $ = $.
	\item Click the \textbf{Car Lease Payments} worksheet tab.
	\item Click cell \textsf{B6} in the \textbf{Car Lease Payments} worksheet.
	\item Type an asterisk $ * $ for multiplication.
	\item Type the number $ 12 $. This multiplies the monthly lease payments by $ 12 $ to calculate the total payments required for the year.
	\item Press the \keystroke{Enter} key on your keyboard. The value of multiplying the monthly lease payments by $ 12 $ is now displayed on the \textbf{Budget Summary} worksheet.
\end{enumerate}

Figure \ref{02:fig37} shows the results of creating formulas that reference cell locations in the \textbf{Mortgage Payments} and \textbf{Car Lease Payments} worksheets.

\begin{figure}[H]
	\centering
	\includegraphics[width=\maxwidth{.95\linewidth}]{gfx/ch02_fig37}
	\caption{Formulas Referencing Cells in Mortgage Payments and Car Lease Payments Worksheets}
	\label{02:fig37}
\end{figure}

We can now add other formulas and functions to the \textbf{Budget Summary} worksheet that can calculate the difference between the total spend dollars vs. the total net income in cell \textsf{D2}. The following steps explain how this is accomplished.

\begin{enumerate}
	\item Click cell \textsf{D6} in the \textbf{Budget Summary} worksheet.
	\item Type an equal sign $ = $.
	\item Type the function name SUM followed by an open parenthesis $ ( $.
	\item Highlight the range \textsf{C3:C5}.
	\item Type a closing parenthesis $ ) $ and press the \keystroke{Enter} key on your keyboard or simply press the \keystroke{Enter} key to close the function. The total for all annual expenses now appears on the worksheet.
	\item Click cell \textsf{D7} on the \textbf{Budget Summary} worksheet. You will enter a formula to calculate Net Change in Cash in this cell.
	\item Type an equal sign $ = $.
	\item Click cell \textsf{D2}.
	\item Type a minus sign $ - $ and then click cell \textsf{D6}.
	\item Press the \keystroke{Enter} key on your keyboard. This formula produces an output of $ \$1,942 $, indicating our income is greater than our total expenses.
\end{enumerate}

Figure \ref{02:fig38} shows the results of the formulas that were added to the \textbf{Budget Summary} worksheet. The output for the formula in cell \textsf{D7} shows that the net income exceeds total planned expenses by $ \$1,942 $. Overall, having your income exceed your total expenses is a good thing because it allows you to save money for future spending needs or unexpected events.

\begin{figure}[H]
	\centering
	\includegraphics[width=\maxwidth{.95\linewidth}]{gfx/ch02_fig38}
	\caption{Formulas Added to Show Income Is Greater Than Expenses}
	\label{02:fig38}
\end{figure}

We can now add a few formulas that calculate both the spending rate and the savings rate as a percentage of net income. These formulas require the use of absolute references, which were covered earlier in this chapter. The following steps explain how to add these formulas.

\begin{enumerate}
	\item Click cell \textsf{E6} in the \textbf{Budget Summary} worksheet.
	\item Type an equal sign $ = $.
	\item Click cell \textsf{D6}.
	\item Type a forward slash $ / $ for division and then click \textsf{D2}.
	\item Press the \keystroke{F4} key on your keyboard. This adds an absolute reference to cell \textsf{D2}.
	\item Press the \keystroke{Enter} key. The result of the formula shows that total expenses consume $ 94.1\% $ of our net income.
	\item Click cell \textsf{E6}.
	\item Place the mouse pointer over the Auto Fill Handle.
	\item When the mouse pointer turns to a black plus sign, left click and drag down to cell \textsf{E7}. This copies and pastes the formula into cell \textsf{E7}.
	\item Save the \textbf{CH2 Personal Budget} file.
	\item Compare your work with the self-check answer key (found in the Course Files) and then submit the \textbf{CH2 Personal Budget} workbook as directed by your instructor.
	\item Close the \textbf{CH2 Personal Budget} file before moving on to \ref{ch02:preparing_to_print}: \nameref{ch02:preparing_to_print}.
\end{enumerate}

Figure \ref{02:fig39} shows the output of the formulas calculating the spending rate and savings rate as a percentage of net income. The absolute reference shown for cell \textsf{D2} prevents the cell from changing when the formula is copied from cell \textsf{E6} and pasted into cell \textsf{E7}. The results of the formula show that our current budget allows for a savings rate of $ 5.9\% $. This is a fairly good savings rate.

\begin{figure}[H]
	\centering
	\includegraphics[width=\maxwidth{.95\linewidth}]{gfx/ch02_fig39}
	\caption{Calculating the Savings Rate}
	\label{02:fig39}
\end{figure}

\begin{center}
	\begin{tkwbox}{Key Take-Aways}
		\textbf{Save}
		\\
		\begin{itemize}
			\setlength{\itemsep}{0pt}
			\setlength{\parskip}{0pt}
			\setlength{\parsep}{0pt}
			
			\item The PMT function can be used to calculate the monthly mortgage payments for a house or the monthly lease payments for a car.
			\item When using the PMT function, each argument must be separated by a comma.
			\item To calculate the monthly payment for a loan using the PMT function, the Rate and Nper arguments must be defined in terms of months. The Rate should be divided by $ 12 $ to convert it from an annual rate to a monthly rate. The Nper should be multiplied by $ 12 $ to convert the term of the loan from years to months.
			\item The PMT function produces a negative output if the Pv argument is not preceded by a minus sign. For the purposes of this textbook, a minus sign will be entered before the PV argument in the PMT dialog box.
			
		\end{itemize}
	\end{tkwbox}
\end{center}

\section{Preparing to Print}\label{ch02:preparing_to_print}

\begin{center}
	\begin{objbox}{Learning Objectives}
		\begin{itemize}
			\setlength{\itemsep}{0pt}
			\setlength{\parskip}{0pt}
			\setlength{\parsep}{0pt}
			
			\item Review and learn new cell formatting techniques.
			\item Understand how to modify page scaling and margins.
			\item Create custom headers and footers to automatically update information.

		\end{itemize}
	\end{objbox}
\end{center}

In this section, we will review some of the formatting techniques covered in Chapter \ref{ch01:fundamental_skills}, \nameref{ch01:fundamental_skills}, as well as learn some new techniques. We will also preview a two-page worksheet and set page setup options to present the data in a professional manner. A new data file will be used in this section.

\begin{figure}[H]
	\centering
	\includegraphics[width=\maxwidth{.95\linewidth}]{gfx/ch02_fig40}
	\caption{Finished Prepare to Print Worksheet in Print Preview}
	\label{02:fig40}
\end{figure}

\textit{Data File: CH2 PTP Data}

You have been given sales data that needs to be formatted in a professional manner. This worksheet will be printed and presented to investors, so it needs to be prepared for printing as well. Figure \ref{02:fig40} shows how the finished worksheet will appear in Print Preview.

\begin{enumerate}
	\item Open the Data file named \textbf{CH2 PTP Data} and use the File/Save As command to save it with the new name \textbf{CH2 Sales Data}.
	\item To change the font of the entire worksheet, click the Select All button in the top left corner of the worksheet grid (see Figure \ref{02:fig41}).
	\item Change the font to Calibri, Size 12.
\end{enumerate}

\begin{figure}[H]
	\centering
	\includegraphics[width=\maxwidth{.95\linewidth}]{gfx/ch02_fig41}
	\caption{Select All button}
	\label{02:fig41}
\end{figure}

Using the skills learned in Chapter \ref{ch01:fundamental_skills}, \nameref{ch01:fundamental_skills}, make the following formatting changes.

\begin{enumerate}
	\item \textsf{A1:H1} --- Merge and Center; format text as bold and apply a font color and size of your choice
	\item \textsf{A2:H2} --- Merge and Center; format text as bold and italic, apply a font color of your choice
	\item \textsf{A5:H5} --- Apply a dark fill color; format text as white and bold
	\item \textsf{C5:H5} --- Center align
	\item \textsf{A15:H15} --- Apply Top Border to the cells; format text as bold
	\item \textsf{C6:H6} and \textsf{C15:H15} --- Apply Accounting Number format with $ 0 $ decimal places
	\item \textsf{C7:H14} --- Apply Comma style with $ 0 $ decimal places
	\item Highlight \textsf{A6:A14} (salespeople's names) and click the Increase Indent button in the Alignment group on the Home ribbon (see Figure \ref{02:fig42}). This will indent the text from the cell border.
\end{enumerate}

\begin{figure}[H]
	\centering
	\includegraphics[width=\maxwidth{.95\linewidth}]{gfx/ch02_fig42}
	\caption{Increase Indent button}
	\label{02:fig42}
\end{figure}

\subsection{Using Page Setup Options}

Once the worksheet is professionally formatted, you need to look in Print Preview to see how the pages will print.

\begin{enumerate}
	\item With the \textbf{CH2 Sales Data} file still open, go to Backstage View by clicking the File tab on the ribbon. Select \textit{Print} from the menu. Notice that the worksheet is currently printing on two pages, with the page breaking between the April and May columns. To fix this problem, you will first change the left and right margins while still in Print Preview
	\item Click the Margins drop-down arrow in the Settings section (see Figure \ref{02:fig43})
	\item Select \textit{Custom Margins…} at the bottom of the list.
	\item Type in $ 0.5 $ for the Left Margin and $ 0.5 $ for the Right Margin.
	\item Click OK. Changing the margins brought the May column onto the same page, but the June column is still on a separate page. Next you will use Page Scaling to fix this while still in Print Preview.
	\item Click the Scaling drop-down arrow in the Settings section (Figure \ref{02:fig43}).
	\item Select \textit{Fit All Columns on One Page}.
	\item Exit Backstage View.
\end{enumerate}

\begin{figure}[H]
	\centering
	\includegraphics[width=\maxwidth{.95\linewidth}]{gfx/ch02_fig43}
	\caption{Settings section of Print Preview}
	\label{02:fig43}
\end{figure}

\subsection{Creating a Footer Using Page Setup}

Now that the entire worksheet is printing on one page, you need to add a footer with information about the date the file was printed along with the filename. In Chapter \ref{ch01:fundamental_skills}, \nameref{ch01:fundamental_skills}, you learned how to create headers and footers using the Insert ribbon. You can also create headers and footers using the Custom Header/Footer dialog box.

\begin{enumerate}
	\item Click the \textbf{Page Layout} tab on the ribbon.
	\item Click the dialog box launcher in the Page Setup group. A window similar to Figure \ref{02:fig44} should appear.
\end{enumerate}

\begin{figure}[H]
	\centering
	\includegraphics[width=\maxwidth{.95\linewidth}]{gfx/ch02_fig44}
	\caption{Page Setup Dialog Box}
	\label{02:fig44}
\end{figure}

\begin{enumerate}[resume]
	\item Click the Header/Footer tab in the Page Setup dialog box.
	\item Click the Custom Footer button. The Footer dialog box should appear (see Figure \ref{02:fig45}).
\end{enumerate}

\begin{figure}[H]
	\centering
	\includegraphics[width=\maxwidth{.95\linewidth}]{gfx/ch02_fig45}
	\caption{Footer Dialog Box}
	\label{02:fig45}
\end{figure}


\begin{enumerate}
	\item Click in the Left section: box and type \textbf{Printed on}.
	\item Making sure to leave a space after the word on, click the Insert Date button.
	\item Click in the Right section: box and type \textbf{Filename:}.
	\item Making sure to leave a space after the colon, click the Insert File Name button.
	\item The Footer dialog box should look like Figure \ref{02:fig46}.
	\item Click the OK button. Click OK again to close the Page Setup dialog box.
	\item Go to Print Preview to see that the current date and file name are displayed in the footer.
	\item Exit Backstage View. Save the \textbf{CH2 Sales Data} file.
	\item Compare your work with the self-check answer key (found in the Course Files) and then submit the \textbf{CH2 Sales Data} workbook as directed by your instructor.
\end{enumerate}

\begin{figure}[H]
	\centering
	\includegraphics[width=\maxwidth{.95\linewidth}]{gfx/ch02_fig46}
	\caption{Completed Custom Footer Dialog Box}
	\label{02:fig46}
\end{figure}

\begin{center}
	\begin{tkwbox}{Key Take-Aways}
		\textbf{Save}
		\\
		\begin{itemize}
			\setlength{\itemsep}{0pt}
			\setlength{\parskip}{0pt}
			\setlength{\parsep}{0pt}
			
			\item It is important to always check your workbooks in Print Preview to ensure that the data is printed in a professional and easy to read manner.
			\item Adjust margins and page scaling as needed to keep columns of data together on one page if possible.
			\item Use headers and footers to display information in the top and bottom margins of the printed worksheet.
			\item Use the Insert buttons to insert changing information, such as dates and file names, instead of typing them in directly.
			
		\end{itemize}
	\end{tkwbox}
\end{center}

\section{Chapter Practice}

\subsection{Financial Plan for a Lawn Care Business}

\textit{Data File: PR2 Data}

Running your own lawn care business can be an excellent way to make money over the summer while on break from college. It can also be a way to supplement your existing income for the purpose of saving money for retirement or for a college fund. However, managing the costs of the business will be critical in order for it to be a profitable venture. In this exercise you will create a simple financial plan for a lawn care business by using the skills covered in this chapter.

\begin{enumerate}
	\item Open the file named \textbf{PR2 Data} and then Save As \textbf{PR2 Lawn Care}.
	
	\item Click cell \textsf{C5} in the \textbf{Annual Plan} worksheet.

	\item Write a formula that calculates the average price per lawn cut. Do \textit{not} use the AVERAGE function. The formula should be the Price per Acre multiplied by the Average Acreage per Customer.
	
	\item Click cell \textsf{C8}.
	
	\item Enter a formula that calculates the total number of lawns that will be cut during the year: \formatfunc{Number of Customers * Frequency of Lawn Cuts per Customer}.
	
	\item Click cell \textsf{D9}.
	
	\item Enter a formula that calculates the total sales for the plan: \formatfunc{Average Price per Cut * Total Lawn Cuts}.
	
	\item Click cell \textsf{F3} in the \textbf{Leases} worksheet. The PMT function will be used to calculate the monthly lease payment for the first item. For many businesses, leasing (or renting) equipment is a more favorable option than purchasing equipment because it requires far less cash. This enables you to begin a business such as a lawn care business without having to put up a lot of money to buy equipment. The PMT function can be entered using the Insert Function button as seen in this chapter, or we can type the PMT function directly into a cell. For this assignment, we will type the function into cell \textsf{F3} using the following instructions.
	
	\item Type \formatfunc{=PMT(}. Define the arguments of the function as follows.

	\begin{enumerate}
		\item \textbf{Rate}: Click cell \textsf{B3}, type a forward slash $ / $ for division, type the number $ 12 $, and type a comma ,. Since we are calculating monthly payments, the annual interest rate must be converted to a monthly interest rate by dividing by $ 12 $.
		
		\item \textbf{Nper}: Click cell \textsf{C3}, type $ *12 $ and then type a comma ,. Similar to the Rate argument, the terms of the lease must be converted to months by multiplying by $ 12 $ since we are calculating monthly payments.
		
		\item \textbf{Pv}: Type a minus sign $ - $, click cell \textsf{D3}, and type a comma ,. Remember that this argument must always be preceded by a minus sign.
		
		\item \textbf{Fv}: Click cell \textsf{E3} (Residual Value) and type a comma ,.
		
		\item \textbf{Type}: Type the number $ 1 $, type a closing parenthesis $ ) $, and press the \keystroke{Enter} key. We will assume the lease payments will be made at the beginning of each month, which requires that this argument be defined with a value of $ 1 $.
	\end{enumerate}

	\item Copy the PMT function in cell \textsf{F3} and paste it into the range \textsf{F4:F6}, or use the Autofill handle.
	
	\item Click cell \textsf{F10} in the \textbf{Leases} worksheet. Enter an Autosum function to calculate the total for the monthly lease payments. Make sure that blank rows (7 through 9) were included in the range for the SUM function. If other items are added to the worksheet, they will be included in the output of the SUM function.
	
	\item Highlight the range \textsf{A2:F6} on the \textbf{Completed Lawn Care Leases Worksheet} worksheet. The data in this range will be sorted, first by Interest Rate and then by Price. Click the Sort button in the Data tab of the Ribbon. In the Sort dialog box, select the \textit{Interest Rate} option in the ``Sort by'' drop-down box. Select \textit{Largest to Smallest} for the sort order. Then, click the Add Level button on the Sort dialog box. Select the \textit{Price} option in the ``Then by'' drop-down box. Select \textit{Largest to Smallest} for the sort order. Click the OK button in the Sort dialog box.
	
	\item Click cell \textsf{B11} on the \textbf{Annual Plan} worksheet. The monthly lease payments that are calculated in the Lease worksheet will be displayed in this cell.
	
	\item Type an equal sign $ = $. Click the \textbf{Leases} worksheet, click cell \textsf{F10}, and press the \keystroke{Enter} key.
	
	\item Click cell \textsf{C12} on the \textbf{Annual Plan} worksheet. Create a formula that calculates the annual lease payments. This should be the \formatfunc{Monthly Lease Payments $ * 12 $}.
	
	\item Click cell \textsf{C14} on the \textbf{Annual Plan} worksheet. Create a formula that calculates the Total Lawn \& Equipment Expenses (\formatfunc{Lawn \& Equipment Expenses per Cut * Total Lawn Cuts}).
	
	\item Click cell \textsf{D16} on the \textbf{Annual Plan} worksheet. Enter a SUM function that adds the Expenses for the business in column C. Make sure to add the Expenses only (not the Sales Plan information).
	
	\item Click cell \textsf{D17} on the \textbf{Annual Plan} worksheet. Enter a formula that calculates the annual profit (Operating Income) for the business. This should be the \formatfunc{Total Sales $ – $ Total Expenses}.
	
	\item Format all cells that contain money amounts in the \textbf{Annual Plan} worksheet for Accounting Number Format (\$) with no decimals.
	
	\item Click cell \textsf{B10} on the \textbf{Investments} worksheet. Enter a COUNT function that counts the number of investments that currently have a balance in column B. Make sure that the additional blank rows in rows 6 through 8 are included in the range for this function. The function output will automatically change if any new investments are added to the worksheet. It is important to note, however, that the Total in cell \textsf{B9} should not be included in the Count range.
	
	\item Click cell \textsf{D3} on the \textbf{Investments} worksheet.
	
	\item Type an equal sign $ = $. Click the \textbf{Annual Plan} worksheet. Click cell \textsf{D17} and type a forward slash $ / $ for division. Click the \textbf{Investments} worksheet. Click cell \textsf{B10} and press the \keystroke{Enter} key. This formula divides the profit calculated on the Annual Plan worksheet by the number of investments in the \textbf{Investments} worksheet. We will assume that the profits from this business will be invested evenly among the funds listed in Column A of the Completed Lawn Care Leases Worksheet worksheet.
	
	\item Before copying and pasting the formula created in cell \textsf{D3}, absolute references must be added to the cell locations in the formula. Edit the formula in cell \textsf{D3} on the Completed Lawn Care Leases Worksheet worksheet so that cells \textsf{D17} and \textsf{B10} are absolute. The formula in cell D3 should be: \formatfunc{='Annual Plan'!\$D\$17/ Investments!\$B\$10}. When you copy the formula in cell \textsf{D3} down, the cell references will not change, because they are absolute. The formula will continue to divide the Operating Income in cell \textsf{D17} of the Annual Plan by the Number of Investments in cell \textsf{B10} of the \textbf{Investments} sheet.
	
	\item Copy cell \textsf{D3} and paste it into cells \textsf{D4} and \textsf{D5} or use the Auto Fill handle to copy down.
	
	\item Click cell \textsf{B9} on the \textbf{Investments} worksheet. Enter a SUM function that adds the current balance for all investments in column B. Make sure that blank rows (rows 6 through 8) are added to the range for the function so additional investments will automatically be included in the Autosum function output.
	
	\item Copy the SUM function in cell \textsf{B9} and paste it into cell \textsf{D9}.
	
	\item Format the \textbf{Investments and Leases} sheets appropriately for Accounting Format. This should include \$ signs on the top row and total row for money amounts, and comma style in the middle rows. In the Investments sheet, apply Comma Style with $ 0 $ decimals to the ranges \textsf{B4:B5} and \textsf{D4:D5}. In the Leases worksheet, apply Accounting Number Format (\$) with two decimals to the range \textsf{D3:F3} and \textsf{F10}. Apply comma format with two decimals to the range \textsf{D4:F9}. Double check that your formatting matches Figures \ref{02:fig47}, \ref{02:fig48}, and \ref{02:fig49}.
	
	\item Save the \textbf{PR2 Lawn Care} workbook.
	
	\item Compare your work with the self-check answer key (found in the Course Files) and then submit the \textbf{PR2 Lawn Care} workbook as directed by your instructor.
\end{enumerate}

\begin{figure}[H]
	\centering
	\includegraphics[width=\maxwidth{.95\linewidth}]{gfx/ch02_fig47}
	\caption{Completed Lawn Care Annual Plan Worksheet}
	\label{02:fig47}
\end{figure}

\begin{figure}[H]
	\centering
	\includegraphics[width=\maxwidth{.95\linewidth}]{gfx/ch02_fig48}
	\caption{Completed Lawn Care Investments Worksheet}
	\label{02:fig48}
\end{figure}

\begin{figure}[H]
	\centering
	\includegraphics[width=\maxwidth{.95\linewidth}]{gfx/ch02_fig49}
	\caption{Completed Lawn Care Leases Worksheet}
	\label{02:fig49}
\end{figure}

\section{Chapter Scored}

\subsection{Hotel Occupancy and Expenses}

\textit{Data File: SC2 Data}

The hotel management industry presents a wide variety of career opportunities. These range from running a bed and breakfast to a management position at a large hotel. No matter what hotel management career you choose to pursue, understanding hotel occupancy and costs are critical to running a successful operation. This exercise examines the occupancy rate and expenses of a small hotel.

\begin{enumerate}
	\item Open the file named \textbf{SC2 Data} and then Save As \textbf{SC2 Hotel}.
	
	\item Enter a formula in cell \textsf{C5} on the \textbf{Occupancy} worksheet to calculate the January capacity for the hotel. The capacity shows how many people the hotel can hold during the month. It is calculated by first multiplying the occupants per room by the number of rooms in the hotel. This result is then multiplied by the number of days in the month (cell \textsf{B5} for January). Create this formula using absolute references so that the appropriate cells do not change when the formula is pasted throughout column C. Hint: two of the cells in the formula need to be absolute references.
	
	\item Copy the formula in cell \textsf{C5} and paste it into the range \textsf{C6:C16}. Use a paste method that does not remove the border at the bottom of cell \textsf{C16}.
	
	\item Enter a formula in cell \textsf{E5} on the \textbf{Occupancy} worksheet to calculate the Percent Occupied of the hotel (this statistic shows what percentage of the hotel is full or occupied). Your formula should divide the Actual Occupancy by the Hotel Capacity. Then copy and paste the formula into the range \textsf{E6:E16}. Use a paste method that does not remove the border at the bottom of cell \textsf{E16}. Format the results in \textsf{E5:E16} as percentages with two decimal places.
	
	\item Enter a function in cell \textsf{C17} on the \textbf{Occupancy} worksheet that sums the values in the range \textsf{C5:C16}. Copy the function and paste it into cell \textsf{D17}.
	
	\item Copy the formula in cell \textsf{E16} and paste it into cell \textsf{E17}. Make sure cell \textsf{E17} is formatted as a percentage with two decimals and bold.
	
	\item On the \textbf{Statistics} worksheet, enter a function into cell \textsf{B3} that shows the highest value (Max) in the range \textsf{D5:D16} in the Actual Occupancy column on the \textbf{Occupancy} worksheet.
	
	\item On the \textbf{Statistics} worksheet, enter a function into cell \textsf{B4} that shows the lowest value (Min) in the range \textsf{D5:D16} in the Actual Occupancy column on the \textbf{Occupancy} worksheet.
	
	\item On the \textbf{Statistics} worksheet, enter a function into cell \textsf{B5} that shows the average value in the range \textsf{D5:D16} in the Actual Occupancy column on the \textbf{Occupancy} worksheet.
	
	\item Use the Auto Fill handle to copy the formulas in the range \textsf{B3:B5} to the range \textsf{C3:C5}.
	
	\item Format the range \textsf{B3:B5} for comma format with zero decimal places. Format cells \textsf{C3:C5} as percentages with two decimal places.
	
	\item The hotel is considering buying or leasing a car to shuttle customers to and from the airport. You hope to keep the monthly payment under $ \$400 $ and will determine whether leasing or buying will meet that goal. On the \textbf{Lease or Buy} worksheet, type the terms in Figure \ref{02:fig50} for the purchase vs. lease of the car. Make sure that dollar amounts and percentages are formatted to match Figure \ref{02:fig50}.
	\end{enumerate}

\begin{figure}[H]
	\centering
	\includegraphics[width=\maxwidth{.95\linewidth}]{gfx/ch02_fig50}
	\caption{Terms for the Purchase vs. Lease of the Car}
	\label{02:fig50}
\end{figure}

\begin{enumerate}[resume]
	\item In cell \textsf{B8} create a PMT function to calculate the Monthly Payment if you purchase the car. Make sure the arguments in the PMT function are converted into months and that the Monthly Payment is a positive number.
	
	\item In cell \textsf{C8} create a PMT function to calculate the Monthly Payment if you lease the car. The car will have a residual value of $ \$15,000 $ when the lease is over. Assume that payments are made at the end of the month.
	
	\item Format the Monthly Payments in \textsf{B8:C8} for Accounting Number Format with two decimals.
	
	\item Select the range \textsf{A4:A8} and click the Increase Indent button once to indent the labels in column A.
	
	\item From the Page Layout tab, access the Page Setup dialog box launcher and center the \textbf{Lease or Buy} worksheet horizontally on the page.
	
	\item Insert a footer on the \textbf{Lease or Buy} worksheet. Insert the date (use the Insert Date button) in the left section of the footer. Insert the File Name (use the Insert File Name button) in the right section of the footer.
	
	\item Save the \textbf{SC2 Hotel} workbook.
	
	\item Submit the \textbf{SC2 Hotel} workbook as directed by your instructor.
\end{enumerate}