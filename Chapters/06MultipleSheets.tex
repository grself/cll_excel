%*****************************************
\chapter{Multiple Sheets}\label{ch06:multiple_sheets}
%*****************************************

                         CHAPTER 6 - MULTIPLE SHEET FILES



Excel workbooks often contain a large amount of data, and worksheets can quickly become
overwhelming.When one worksheet becomes cumbersome, data can be broken out into smaller
subsets and placed in separate worksheets within the same Excel file. Separating out spreadsheet data
into smaller pieces can lead to better data organization within a file and increase its ease of use.When
a retail company needs to track overall sales, as well as, individual store sales, it makes sense to place
each store’s sales data in a separate sheet within a file. Adding a summary sheet that sums across all
the sheets will allow for total company sales data in the same file! This chapter will show how to set
up a workbook to make multi-sheet formulas quick and easy.

Other examples of when multiple sheets make the most sense are when you are comparing regional
data, data for a salesforce and wish to evaluate individual salesperson performance along with overall
sales, and data over a period of time where sheets can be broken out by year or by month. When
comparing data across several sheets, it is essential that all the sheets are laid out in the same way.
To facilitate this, a template can be used. A template is the basic pattern for each new sheet that can
be used repeatedly to make sure each new sheet has the same setup, formatting, formulas, etc. as the
existing sheets in a file. In this chapter, we will use both pre-made, Microsoft Excel templates, as well
as, ones we will create from scratch to meet the specific needs of our work.


Attribution
Chapter 6 – Multiple Sheet Files by Diane Shingledecker, Portland Community College is licensed
under CC BY 4.0
6.1 MULTIPLE SHEET BASICS




Learning Objectives


1. Navigating through a multiple sheet file.
2. Adding, deleting, copying, and moving sheets.
3. Grouping and ungrouping sheets.



Throughout our textbook, we have worked with Excel files that have included multiple sheets.
Depending on the version of Excel you are using, a new Excel file starts with several sheets. In this
chapter, we will be working with a personal budget file that contains income and expenses for an
entire year. Our file contains a sheet for each month of the year, as well as, a Summary sheet that
will add all twelve monthly sheets of data together. To begin with, we’ll get comfortable with moving
through the sheets, organizing them, and making sure that all twelve monthly sheets are consistent.
Figure 6.1 shows the January sheet in the Personal Budget file along with all the sheet tabs along the
bottom of the window.




BEGINNING EXCEL 303
Figure 6.1 Personal Budget File


NAVIGATING THROUGH A MULTIPLE SHEET FILE

Download Data file: CH6 Data

1. Open the data file CH6 Data and save the file to your computer as CH6 Personal Budget. Notice
that the file has a Expenses Summary sheet at the far left followed by monthly sheets.

2. Click on the different sheets at the bottom of the screen to move through the sheets. Notice that the
Expenses Summary sheet is formatted differently from the monthly sheets. Notice also that all the
monthly sheets are identical in layout and format.

3. Take a second look at the months at the end of the year. Not all the data has been entered for
September through October… and there is no sheet for December. We will enter the remaining data
and add an additional sheet for December.

4. Add the following data in the September, October, and November sheets:




304 NOREEN BROWN, BARBARA LAVE, JULIE ROMEY, MARY SCHATZ, DIANE SHINGLEDECKER
September October November
Power        $135     $135    $135
Water        $30      $30     $30
Groceries    $300     $325    $400
Miscellaneous $100    $50     $100
Bonus
Freelance    $500             $150
Other                 $100


COPYING A SHEET

1. To make a December sheet, we are going to copy the November sheet.
2. Point your mouse at the November sheet tab at the bottom of the screen.
3. Hold down your left mouse button and then press and hold down the CTRL key.
4. At this point, you will see a black down-pointing arrow to the left of the November sheet tab
and your mouse cursor will become a small piece of paper with a plus sign on it.
5. Drag your mouse to the right (still holding down the left-mouse button and the CTRL key) until
the black down-pointing arrow is to the right of the November sheet tab.
6. Let go of the mouse button and then the CTRL key. You should now have a November (2) sheet
to the right of the November sheet as shown in Figure 6.2.




BEGINNING EXCEL 305
Figure 6.2 Additional November Sheet


Next, we’ll update the November (2) sheet to turn it into our December sheet.

1.   Right-click on the November (2) sheet name at the bottom of the screen and choose Rename.
2.   Type “December” and press Enter.
3.   Click on the December sheet.
4.   Click on B1 and change “November” to “December”.
5.   Make the following data changes:

◦ Miscellaneous: $300
◦ Bonus: $250 (it’s the holidays!)
◦ Freelance: delete amount

6.   Save your work.
7.   Point your mouse at the December sheet tab at the bottom of the screen.
8.   Hold down your left mouse button and then press and hold down the CTRL key.
9.   Drag your mouse to the right (still holding down the left-mouse button and the CTRL key) until
the black down-pointing arrow is to the right of the December sheet.

306 NOREEN BROWN, BARBARA LAVE, JULIE ROMEY, MARY SCHATZ, DIANE SHINGLEDECKER
10. Let go of the mouse button and then the CTRL key. You should now have a December (2) sheet
to the right of the December sheet.
11. Rename the December(2) sheet Practice.


Skill Refresher


Copying a Sheet

1. Point your mouse at the sheet you want to copy at the bottom of the screen.
2. Hold down your left mouse button and then press and hold down the CTRL key.
3. Drag your mouse to the right (still holding down the left-mouse button and the CTRL key) until the black
down-pointing arrow is to the right of your existing sheet.
4. Let go of the mouse button and then the CTRL key. You should now have a Sheetname (2) to the right of
the original sheet.
5. Rename the Sheetname (2) sheet as desired.



MOVING AND DELETING SHEETS

Sometimes your sheets do not end up in the right order, and you need to move them in order to fix
this. Let’s try moving our “Practice” sheet to see how this is done.

1. Point to the Practice sheet and hold down your left mouse button.
2. Notice this time that there is still a black arrow to the left of the Practice sheet, but the piece of
paper is blank. It does not have a plus sign (+) because we are moving, instead of copying, the
sheet.
3. Left-drag the mouse to the right until the black arrow marker is between the October and
November sheets.
4. Release the mouse button.
5. Try moving the Practice sheet back to the right of the December sheet.

Since our Practice sheet is not a sheet we will need in our Budget file, we’ll go ahead and delete it now.

1. Right-click on the Practice sheet tab at the bottom of the screen.
2. Click Delete. Figure 6.3 shows the warning message box that will appear on your screen. Your
message box might look slightly different depending on the version of Excel you are using. It is
important to note that you cannot Undo once you delete a sheet!




Figure 6.3 Warning Message Box

BEGINNING EXCEL 307
3. Click Delete.

GROUPING AND UNGROUPING SHEETS

Take a look at our monthly sheets again. Notice that there is a place in each of these sheets to calculate
three pieces of Summary data: Income, Expenses, and Balance; but there aren’t any formulas in these
cells. There is also a place for the % of Income Spent, but we need a formula in I6:I7 to calculate this.
If we entered these formulas individually in each of the 12 month sheets, it would take a long time!
Because this task would be very repetitive, it would also be fairly likely that we would make some
mistakes along the way entering the same formulas over and over again. By grouping all the month
sheets together, we can enter each of the formulas once and have them appear in all the sheets.

1. Click on the January sheet to make it active.
2. Hold the SHIFT key down and click on the December sheet.

Now all 12 sheets should be selected. You can tell this in two ways: the sheet tabs that have been
selected are now bold at the bottom of your screen. Notice in the Title bar at the top of the screen the
word [Group] added to the end of the title. You can see both of these in Figure 6.4.




Figure 6.4 Grouped Sheets


IT IS IMPORTANT TO REMEMBER THAT ANY CHANGES WE MAKE TO THE JANUARY
SHEET WILL BE MADE TO ALL THE SHEETS!! This is a very good thing when we want to make
changes to all the sheets at once, but we need to be sure to ungroup them when we’re done making
these changes.Let’s go ahead and add the formulas to all twelve of the sheets at once:

1. Click in F11 in the January grouped sheet.
2. Enter the formula =SUM(F5:F8).

308 NOREEN BROWN, BARBARA LAVE, JULIE ROMEY, MARY SCHATZ, DIANE SHINGLEDECKER
3. In F12, enter the formula =SUM(C5:C13).
4. In F13, subtract Expenses from Income. In the January sheet, your balance should be $690.
HINT: if your answer is negative, you subtracted Income from Expenses.
5. Click on I6. (I6 and I7 are formatted and merged together – this is fine.)
6. Enter a formula that divides Expenses (F12) by Income (F11). Your answer will show as a
percentage since this cell has already been formatted to do this. HINT: If you percentage is
greater than 100%, you have your numbers reversed.

Notice that a data bar was set up in I5 to visually show the income spent. Do you remember how to
do this from earlier in our textbook? Your January sheet should now look like Figure 6.5.




Figure 6.5 January Sheet with Formulas


7. Now that we are done making changes to all the monthly sheets at once, we need to ungroup them.
Right-click on one of the grouped sheets and choose Ungroup Sheets.

Notice the sheets tabs are no longer bold and the word [Group] is no longer in the title bar.

8. Click on several of the month sheets to see that all the formulas have been added.

9. Click on the December sheet. Your sheet should now look like Figure 6.6.




BEGINNING EXCEL 309
Figure 6.6 December Sheet with Formulas


1. Take a look at the Notes in the September sheet. It says that the rent was raised in September, so
we need to cancel our Gym Membership and show $0 for the Gym amount in October,
November, and December.
2. Group the October, November, and December sheets. If you do this successfully, these three
sheet names should be bold and the word [Group] will appear in the Title bar.
3. Click on C13 and change the amount to $0. Press Enter.
4. Ungroup the sheets.The balances should be: October $605, November $530, and December
$430.


Skill Refresher


To Group Sheets:
Click on the leftmost sheet you want to group; then hold the SHIFT key down and click on the rightmost sheet you want to
group.

To Ungroup Sheets:
Right-click on one of the grouped sheets and choose Ungroup Sheets.




Key Takeaways


• You can easily move, copy, delete, and rename sheets in your Excel file.



310 NOREEN BROWN, BARBARA LAVE, JULIE ROMEY, MARY SCHATZ, DIANE SHINGLEDECKER
• Grouping sheets allows you to change a group of identically formatted sheets at the same time.



ATTRIBUTION

“6.1 Multiple Sheet Basics” by Diane Shingledecker, Portland Community College is licensed under
CC BY 4.0




BEGINNING EXCEL 311
6.2 FORMULAS WITH 3-D REFERENCES




Learning Objectives


• Entering formulas that reference another sheet.

• Using the SUM function to add up multiple sheets.



The Summary sheet in many multiple sheet workbooks is utilized to present totaled information
from the other sheets in the file. This is done to give a quick synopsis of all the other sheets in one
convenient location. For this reason, the Summary sheet is usually the first sheet in multiple sheet
files. Summary sheets “pull” data from the other sheets using three-dimensional (3-D) cell references.
In order to distinguish between A3 in the Summary sheet, A3 in the January sheet, A3 in the February
sheet, etc.; a 3-D cell reference includes the sheet name along with the cell reference. The syntax to
reference a cell in a different sheet is =SheetName!CellRange. So, the cell reference for A15 in the
March sheet would be =March!A15.

Let’s start working on our summary sheet by trying out some 3-D formulas:

1. Click on the Expenses Summary sheet tab at the bottom of the screen.
2. Click on C5 and enter the formula =January!C5. Press Enter.
This will get the amount $700 from cell C5 in the January sheet.
3. Delete the formula in C5 in the Expenses Summary sheet.
4. This time, click on C5 and type =. Then click on the January sheet, and then click on C5.
5. Press Enter. This will put the same formula, =January!C5, in cell C5 in the Expenses Summary
sheet and will return the value $700.
6. In cell C6 in the Expenses Summary sheet, try entering a formula for the Power amount in the
April sheet. You should get $135 as the Power amount.
7. Delete the formulas in cells C5 and C6 in the Expenses Summary sheet.

For the Annual Amounts in C5:C13 in the Expenses Summary sheet, we don’t need the amount from
a single month’s sheets; instead, we need the sum of all the entries in all the monthly sheets. So, we
need to sum three-dimensionally through all twelve month sheets. Here’s a helpful hint on the steps
you need to follow to add through multiple sheets:



312 NOREEN BROWN, BARBARA LAVE, JULIE ROMEY, MARY SCHATZ, DIANE SHINGLEDECKER
Skill Refresher


To SUM across sheets:

1. Click on the cell where you want the 3-D SUM to appear.
2. Type =SUM(
3. Click on the leftmost sheet in the group of sheets you want to sum.
4. Hold the SHIFT key down and click on the rightmost sheet in the group of sheets you want to sum.
5. Click on the cell in the sheet you’re in that you want to sum.
6. Press ENTER.



Let’s try adding up all the monthly amounts in our Expenses Summary sheet:

1.    Click in C5 in the Expenses Summary sheet.
2.    Type =SUM(. (Make sure to type the open parentheses!)
3.    Click on the January sheet.
4.    Hold the SHIFT key down and click on the December sheet.
5.    Click on C5 again and press ENTER. Cell C5 should display the sum amount of $8,400.
6.    Click on C5 in the Expenses Summary sheet. In the formula bar, you should see the following
formula: =SUM( January:December!C5). This means SUM C5 in the sheets January through
December.
7.    Let’s try another 3-D SUM together. Click on C6.
8.    Type =SUM(. (Make sure to type the open parentheses!)
9.    Click on the January sheet.
10.    Hold the SHIFT key down and click on the December sheet.
11.    Click on C6 again and press ENTER. Cell C6 should now display the sum amount of $1,610.
12.    Click on C6 in the Expenses Summary sheet. In the formula bar, you should see the following
formula: =SUM( January:December!C6).

If you feel comfortable with these 3-D formulas, you can copy C6 down through C13 to fill in the
rest of the formulas. If you’re not quite comfortable yet, keep practicing the above steps to add 3-D
formulas to cells C7:C13. When you’re done, your Expenses Summary sheet should match Figure 6.7.




BEGINNING EXCEL 313
Figure 6.7 Complete Expenses Summary Formulas




While our 3-D formulas are complete in the Expenses Summary sheet, our summary feels like it is
lacking something. Let’s add a visual representation of our summary numbers to the sheet.

1. Highlight cells B5:C13 in the Expenses Summary sheet.
2. Click on Pie Chart in the Insert tab in the ribbon and select the 2-D pie.
3. Move and resize the pie chart so that it fills cells D3:J15.
4. Delete the chart title.
5. Move the legend to the right side of the chart. Resize the legend as needed.
6. Add percentage data labels to the pie slices. Format the data labels to be bold with white font
color. Your complete Expenses Summary sheet should look like Figure 6.8 below.
7. Save your file. If you’re printing your assignment at this point, print ONLY the Summary sheet
in regular and formula view. Close your file.




314 NOREEN BROWN, BARBARA LAVE, JULIE ROMEY, MARY SCHATZ, DIANE SHINGLEDECKER
Figure 6.8 Completed Expenses Summary Sheet




Skill Refresher


3-D References in Formulas

To reference a cell in another sheet, use the formula syntax =SheetName!CellAddress.

To enter a 3-D reference:

1. Click on the cell where you want the formula to appear and type =.
2. Click on the sheet with the cell you want.
3. Click on the cell in the sheet you want and press ENTER.




Key Takeaways


• 3-D references in formulas allow you to use data from one or more sheets on another sheet.



ATTRIBUTION

“6.2 Formulas with 3-D References” by Diane Shingledecker, Portland Community College is licensed
under CC BY 4.0

BEGINNING EXCEL 315
6.3 TEMPLATES




Learning Objectives


1. Use an existing Microsoft Excel template to create a new spreadsheet.
2. Create a custom template to create new spreadsheets.



A template is a predefined pattern for a spreadsheet that has already been created for you. Hundreds
of templates, already created by Microsoft, are available for you to use inside Excel. These templates
are very helpful if you have limited time to get a new task done in Excel, and you don’t know where
to start. Templates do a lot of the work for you! Templates include all the formulas, formatting, etc.
needed in a professional Excel spreadsheet. All that’s left to do is enter the data. Predefined Microsoft
templates include everything from billing statements to blood pressure trackers to business cards.
Categories include: Business, Personal, Industry, Financial Management, Logs, Calculators, and Lists.

Sometimes you need a very specific template that hasn’t already been created by Microsoft. Taking the
time to create your own template will allow you to use this spreadsheet pattern to create files from it
over and over again. If you need to create a new version of a spreadsheet on a regular basis, templates
will make this work much easier. In this chapter we will explore using existing Microsoft templates,
as well as, creating our own templates.

Let’s start by trying out a predefined, Microsoft template:

1.   Click the File tab in the ribbon.
2.   Click New in the Backstage View.
3.   Click in the Search box for Online template.
4.   Type Travel and press ENTER.
5.   Click on the Travel Expense Report and click Create. NOTE: If this template is not available,
ask your instructor which template you should choose.

Your screen should look like Figure 6.9 below. Notice the design, layout, and formulas have already
been set up for you.




316 NOREEN BROWN, BARBARA LAVE, JULIE ROMEY, MARY SCHATZ, DIANE SHINGLEDECKER
Figure 6.9 Travel Expenses Report Template


Try using the template by doing the following:

1. Change the Name to your name.
2. Change the Department to CAS.
3. Press CTRL+~ to see where the formulas are in the sheet. Working in the formula view helps
you see where the formulas are, so you won’t delete them.
4. In formula view, carefully delete just the data. Don’t delete any formulas!
5. Press CTRL+~ again to return to Normal view.
6. Enter dates and expenses for a trip of your imagining in the first three rows under the column
headings.
7. Save the completed file as CH6 Travel Expenses. Close the file.


Skills Refresher


To use a Microsoft predefined template:

1. Click on the File tab in the ribbon.
2. Click on New
3. Type the desired template description in the Search box, and press ENTER.



Now let’s shift to creating our own template. Sometimes you create a blank template first and then
create spreadsheets from it. Other times, you have an existing spreadsheet that you realize you need
a template for, so that you can recreate the file with new data over and over again. We’ll turn our
existing CH6 Personal Budget file into a template now, and in the assignments at the end of the
chapter, we’ll start templates from scratch before a filled-in spreadsheet is created.

1. Open your CH6 Personal Budget.xlsx file.
2. Group the month sheets (January through December).
3. Press CTRL+~ to switch to Formula view.

BEGINNING EXCEL 317
4. We only want to delete data from these sheets – not labels or formulas. The only data is in
C5:C13, F5:F8, and in the Notes in H11:J13.
5. Highlight C5:C13 (with all the sheets still grouped) and press DELETE.
6. Highlight F5:F8 (with all the sheets still grouped) and press DELETE.
7. Highlight H11:J13 (with all the sheets still grouped) and press DELETE.
8. Press CTRL+~ to switch back to Normal view.
9. Ungroup the sheets. Look through the sheets to check that only the data has been deleted.
Notice the error message #DIV/O appears in I6:I7 since the data for this formula has been
deleted. Your January sheet should look like Figure 6.10.




Figure 6.10 January Template Sheet


NOTE: There are only formulas and the pie chart in the Expenses Summary sheet, so nothing needs
to be deleted from this sheet to setup your template.

1.   Click the File tab in the ribbon and then click Save As.
2.   Choose the location where you want to save the file.
3.   In the Save as type pull-down list, select Excel Template (*.xltx).
4.   At the top of the screen, double-check that the location you want to save the file to has not
changed. If it has, use the pull-down list to find the location where you want to save your file. BE
CAREFUL HERE! By default, Excel will try to save this to a default template file location on
your hard drive.

318 NOREEN BROWN, BARBARA LAVE, JULIE ROMEY, MARY SCHATZ, DIANE SHINGLEDECKER
5. Type in the file name CH6 Personal Budget Template.xltx. Check your screen carefully with
Figure 6.11. Keep in mind that you may be saving your template file to a different place on your
computer. By default, Excel will save the template to your hard drive assuming you always work
on the same computer.




Figure 6.11 Save As Template


6. Click Save.


Skills Refresher


To save a file you created as a template:

1. Click the File tab in the ribbon and then click Save As.
2. Choose the location where you want to save the file.
3. In the Save as type pull-down list, select Excel Template (*.xltx).
4. At the top of the screen, double-check that the location you want to save the file to has not changed. If it
has, use the pull-down list to find the location where you want to save your file. BE CAREFUL HERE! By
default, Excel will try to save this to a default template file location on your hard drive.
5. Type in the file name



BEGINNING EXCEL 319
6. Click Save.



We are now going to use our new budget template to start a Personal Budget file for 2017. We want
to use the Ch6 Personal Budget Template to create the new file, but we don’t want to overwrite the
template. We want to keep it clean to use to start each new year’s file. To do this, we’ll save the file to
our new 2017 file name before we start filling in any data.

1.   With the CH6 Personal Budget Template open, click the File tab in the ribbon.
2.   Choose Save As and choose the location where you want to save the 2017 version of the file.
3.   Change the Save as Type back to Excel Workbook (*.xlsx).
4.   Enter the File name CH6 2017 Personal Budget. Compare your screen to Figure 6.12.




320 NOREEN BROWN, BARBARA LAVE, JULIE ROMEY, MARY SCHATZ, DIANE SHINGLEDECKER
Figure 6.12 Save As 2017 Budget File


5.   Click Save.
6.   Group all the sheets together including the Expenses Summary sheet.
7.   Click on H1. Type 2017 and press ENTER.
8.   Ungroup the sheets.
9.   Click on the January sheet. Enter the following data in Figure 6.13:


BEGINNING EXCEL 321
Figure 6.13 January 2017 Data


10. Click on the Expenses Summary sheet – the data and the pie chart should show the January
data since that is all the data in the twelve month sheets for now.Your sheet should look like




Figure 6.14.                                                                             Figure 6.14 Expense
Summary Sheet
11. Save your file.


Key Takeaways


• There are many pre-designed templates in Excel developed in Excel that you can use. This will save you the
time and effort of designing and creating these files from scratch.

• You can create your own template files in Excel that you can use over and over again.



ATTRIBUTION

“6.3 Templates” by Diane Shingledecker, Portland Community College is licensed under CC BY 4.0

322 NOREEN BROWN, BARBARA LAVE, JULIE ROMEY, MARY SCHATZ, DIANE SHINGLEDECKER
6.4 PREPARING TO PRINT




Learning Objectives


1. Printing all of the worksheets in a workbook at one time.
2. Preparing multiple worksheets for printing using grouping.



Just like consistency in formatting is important when working with workbooks containing multiple
worksheets with the same type of data, so is consistency in page setup. Now that the Personal Budget
2017 workbook is complete, you are going to prepare it for printing by changing the page orientation
and adding a header. You will also print all 13 worksheets at one time.

APPLYING PAGE SETUP OPTIONS TO GROUPED WORKSHEETS

Data file: Continue with CH6 2017 Personal Budget.

As always, you need to review your workbook in Print Preview before considering it complete. When
you do that with this workbook, you notice that the worksheets are each printing on two pages. You
decide to switch all the worksheets to Landscape orientation to see if that helps. You will also add a
footer with the worksheet name to each of the worksheets.

1. Go to Print Preview. To view all of the worksheets at one time, select Print Entire Workbook in
the first box in the Settings section. You should now have 26 pages to scroll through in Print
Preview. If you were to click the Print button, all of the worksheets would print, not just the
active sheet.
2. Exit Backstage View. You want to change the page orientation of all the sheets, but you cannot
change all of them at one time in Print Preview.
3. Group all of the worksheets together, including the Expenses Summary sheet through the
December sheet.
4. Click on the Page Layout tab on the ribbon, then select Landscape using the Orientation button
in the Page Setup group.
5. Click the Page Setup dialog box launcher arrow in the Page Setup group then click the Header/
Footer tab.
6. Click the Custom Footer button. In the center section, insert the worksheet name using the
Insert Sheet Name button. The Footer dialog box should look like Figure 6.15.
7. Click OK to close the Footer dialog box. Click OK again to close the Page Setup dialog box.
BEGINNING EXCEL 323
Figure 6. 15 Insert Worksheet Name


8. Return to Print Preview to confirm that each worksheet is printing on one page, in landscape
orientation, with the correct worksheet name appearing in the footer.

In Print Preview you notice that the Expenses Summary sheet is not set to print correctly. Part of
the chart is appearing on a second page. You can easily fix this by changing the Scaling, but you only
want to change the scaling of the Expenses Summary sheet, not the entire workbook. If you make
the change in Print Preview while the worksheets are grouped it will change all of the worksheets.

1. Exit Backstage View.
2. Ungroup the worksheets by right-clicking on any of the worksheet tabs and selecting Ungroup
Sheets.
3. If needed, click on the Expenses Summary worksheet tab to make it the active worksheet.




Figure 6.16 Scale to Fit


4. Click on the Page Layout tab on the ribbon and locate the Scale to Fit group of commands. (See
Figure 6.16 above.)
5. Click the drop-down arrow for Width: and select 1 page. This has the same result as selecting Fit
All Columns on One Page in the Scaling setting in Print Preview.
6. Return to Print Preview to confirm that the Expenses Summary worksheet is now printing on
one page only.
7. Exit Backstage View.
8. Save the CH6 2017 Personal Budget workbook.
9. Compare your work with the self-check answer keys (found in the Course Files link) and then
submit the following files as directed by your instructor.

• CH6 Personal Budget
• CH6 Personal Budget Template
• CH6 Travel Expenses
324 NOREEN BROWN, BARBARA LAVE, JULIE ROMEY, MARY SCHATZ, DIANE SHINGLEDECKER
• CH6 2017 Personal Budget


Key Takeaways


• To print all of the worksheets in a workbook at one time select Print Entire Workbook in the Print Settings.

• You can apply page setup options, such as scaling, orientation, and headers/footers, to multiple worksheets
at one time by grouping the worksheets.



ATTRIBUTION

“6.4 Preparing to Print” by Julie Romey, Portland Community College is licensed under CC BY 4.0




BEGINNING EXCEL 325
6.5 CHAPTER PRACTICE




A MULTIPLE SHEET TEMPLATE FOR A SPORTS TEAM

Data file: PR6 Data

You have just gotten a job with the Pacific Northwest Soccer Club, and you quickly realize that there
isn’t a consistent way for all the coaches to keep track of their team statistics. To help with this,
you decide to make a template for Season Stats for each team. Since you are also the coach of the
High Flyers this season, you will need to use the template to enter your team’s statistics into a team
spreadsheet.

1. Open the data file PR6 Data and save the file to your computer as PR6 Pacific NW Sports
Team.
2. Copy the range B11:G22 in the Season Stats sheet to the same range in the Player Stats sheet.
3. Group the sheets and add the following formulas to both sheets:
4. In C22 and D22, you’ll need to count the Xs in rows 12 through 21. To do this, use a COUNTA
formula.
5. In E22 and F22, sum rows 12 through 21.
6. In G12, calculate Goal Percentage by dividing the number of Shots by the number of Goals. This
will display an error message because there are zeros in column F. We don’t want to display
error messages in the file, so an IF statement that tests the value of column F will solve this
problem.
7. Change the formula in G12 with the following three pieces:
8. Test – is F12 greater than zero
9. If the Test is True – divide the number of Goals by the number of Shot
10. If the Test is False – enter a zero
11. Copy G12 down the column through G22. Format these cells as percentages.
12. For an extra challenge, put the “banded row” format back in G12:G22.
13. Ungroup the sheets.
14. Save the file as a template called PR6 Pacific NW Team Template.xltx. Make sure to save your
template to your USB and not the default folder for templates on your hard drive!
15. Make a new file using the PR6 Pacific NW Team Template and save it as PR6 High Flyers.xlsx.
16. In the Season Stats sheet, enter the following data:

◦ D3 – High Flyers
◦ D4 – Fall and the current year (i.e. – Fall 2016)


326 NOREEN BROWN, BARBARA LAVE, JULIE ROMEY, MARY SCHATZ, DIANE SHINGLEDECKER
◦ D5 – Pacific Northwest Soccer

17. Enter your name, phone number, and email address in row 8.
18. Make four copies of the Player Statistics sheet. Rename the player sheets Player 1, Player 2,
Player 3, Player 4, and Player 5.
19. Group the Player sheets. Enter the following formulas:
20. A formula in D4 that points to cell D3 in the Season Stats sheet. Note: Your formula will be
=’Season Stats’!D3:G3 instead of =’Season Stats’!D3 because D3:G3 are merged together.
21. A formula in D5 that points to cell D4 in the Season Stats sheet.
22. A formula in D6 that points to cell D5 in the Season Stats sheet.
23. Ungroup the sheets.
24. Click on the Player 1 sheet. Enter the Player Name: Juan Ramirez. Enter the following data
from Table 1:
Table 1: Player 1 Sheet

Played Started Shots Goals
Game 1   x      x       2     1
Game 2   x      x       3     1
Game 3
Game 4   x
Game 5   x      x       2     0
Game 6   x
Game 7
Game 8   x      x       1     1
Game 9   x      x       4     2
Game 10 x       x       3     3

25. Click on the Player 2 sheet. Enter the Player Name: Zach Johnson. Enter the following data
from Table 2:
Table 2: Player 2 Sheet

Played Started Shots Goals
Game 1   x      x       1     1
Game 2   x      x       2     1
Game 3   x      x       1     1
Game 4   x      x       1     1
Game 5   x      x       2     0
Game 6   x      x       5     2
Game 7   x      x       4     2
Game 8   x      x       1     1
Game 9   x      x       4     1
Game 10 x       x       3     2

26. Click on the Player 3 sheet. Enter the Player Name: Vito Lawrenz. Enter the following data
BEGINNING EXCEL 327
from Table 3:
Table 3: Player 3 Sheet

Played Started Shots Goals
Game 1   x      x        0     0
Game 2   x      x        1     1
Game 3   x      x        2     0
Game 4   x               1     1
Game 5   x      x        2     0
Game 6   x      x        3     1
Game 7   x      x        2     1
Game 8   x      x        1     1
Game 9   x      x        1     1
Game 10 x       x        1     1

27. Make up information for the names and data in the Player 4 and Player 5 sheets.
28. Go to the Season Stats sheet and click on cell C12. Enter a 3-D formula to COUNTA (count
text) in C12 through sheets Player 1 through Player 5. Copy the formula in C12 through D22.
29. Change the formulas in C22 and D22 to SUMs.
30. Click on E12. Enter a 3-D formula to SUM E12 in sheets Player 1 through Player 5. Copy the
formulas through F22.
31. Preview the worksheets in Print Preview. Notice that only part of the data is printing for each
worksheet. This is because a Print Area was incorrectly set when the file was first created. You
need to clear this Print Area for each worksheet individually (modifying print areas cannot be
done on grouped sheets). Exit Backstage View and for each worksheet, click the Print Area
button on the Page Layout tab and select Clear Print Area.
32. Save the PR6 High Flyers workbook.
33. Compare your work with the self-check answer key (found in the Course Files link) and then
submit the PR6 High Flyers workbook and PR6 Pacific NW Team Template as directed by
your instructor.

ATTRIBUTION

” 6.5 Chapter Practice” by Diane Shingledecker, Portland Community College is licensed under CC
BY 4.0




328 NOREEN BROWN, BARBARA LAVE, JULIE ROMEY, MARY SCHATZ, DIANE SHINGLEDECKER
6.6 SCORED ASSESSMENT




A MULTIPLE SHEET TEMPLATE FOR NATIONAL PARKS DATA

Data file: none

You are working in the National Headquarters of the National Parks Department. One of your jobs
this week is to develop a template for individual park use data, along with a Summary of the parks
visitation data. To do this you will need to create a template with summary and individual park sheets,
and then use the template to enter park data for five national parks.

1. Open a blank spreadsheet.
2. Design a professional quality sheet to display individual park data. Include areas to enter the
name of a park and the park statistics college in Table 4 below. Name the sheet Park Data. (Do
not copy the physical layout of Table 4.) Figure 6.17 below is an example of how you could set
this up:




Figure 6.17 Sample setup

3. Make a copy of the Park Data sheet and rename it Summary Park Data. Change any text
needed for the Summary sheet. Make sure both sheets are laid out exactly the same.
4. Save the file as a template called SC6 National Parks Template.xltx. Make sure this gets saved
to your USB and not the default folder on your hard drive! Close the template once it has been
saved.
BEGINNING EXCEL 329
5. Make a new file using the template and save it as SC6 Five National Parks.xlsx.
6. Make four copies of the Park Data sheet. Rename the five Park Data sheets after the five parks
listed in Table 4.

Table 4: 2015 National Park Data

Non-                      Non-
Recreation                Recreation                Concessioner   Concessioner   Tent      RV        Backcountry   Miscellaneous
Park                       Recreation                Recreation
Visitors                  Hours                     Lodging        Camping        Campers   Campers   Campers       Overnight Stays
Visitors                  Hours

Acadia NP     2,811,184    47,100       14,452,151   47,100       0              0              135,000   32,094    1,233         8,343


Blue Ridge
15,054,603   1,942,260    93,977,122   971,136      53,688         0              61,481    33,499    2,101         1,294
PKWY


Crater Lake
614,712      49,600       4,033,484    24,800       34,629         55,596         7,548     0         3,253         0
NP


Yellowstone
4,097,710    1,156,118    82,016,845   711,795      552,940        584,979        104,149   69,830    44,898        11,715
NP


Yosemite
4,150,217    155,081      78,505,877   3,993,223    938,418        0              588,701   284,372   211,966       39,214
NP




1. Enter the park data for each of the parks in each of the five sheets.
2. In the Summary Park Data sheet, create formulas for all the fields that add up the five park
sheets.
3. Preview the entire workbook in Print Preview to ensure that it is printing professionally. Make
any changes needed.
4. Save the SC6 Five National Parks workbook.
5. Submit both the SC6 Five National Parks and SC6 National Parks Template files as directed by
your instructor.

National Park 2015 Data from: https://irma.nps.gov/Stats/SSRSReports

ATTRIBUTION

“6.6 Scored Assessment” by Diane Shingledecker, Portland Community College is licensed under CC
BY 4.0


