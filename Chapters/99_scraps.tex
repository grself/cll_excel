%************************************************************************
\section{Box: Text}
%************************************************************************
% Creates a nice boxed text with a title and main section
\begin{tcolorbox}[colback=blue!5!white,colframe=blue!75!black]
	% Upper half of box: my "title" area
	\textcolor{blue}{\textbf{Interesing Note}}
	% Lower half of the box: the content
	\tcblower
	Whatever.
\end{tcolorbox}

%***************************************************************************
\section{Box: Learning Objectives}
% Relies on ``objbox'' in config file -- brown color
%***************************************************************************
\begin{center}
	\begin{objbox}{Learning Objectives}
		\begin{itemize}
			\setlength{\itemsep}{0pt}
			\setlength{\parskip}{0pt}
			\setlength{\parsep}{0pt}
		
			\item x1.
			\item x2.
			\item x3.
		\end{itemize}
	\end{objbox}
\end{center}

%***************************************************************************
\section{Box: Information}
% Relies on ``infobox'' in config file -- purple color
%***************************************************************************
\begin{center}
	\begin{infobox}{Information}
		\textbf{Columns}
		\\
		\\
		Whatever
	\end{infobox}
\end{center}

%***************************************************************************
\section{Box: Keyboard Shortcuts}
% Relies on ``shtcutbox'' in config file -- cyan color
%***************************************************************************
\begin{center}
	\begin{shtcutbox}{Keyboard Shortcuts}
		\textbf{Save}
		\\
		\begin{itemize}
			\setlength{\itemsep}{0pt}
			\setlength{\parskip}{0pt}
			\setlength{\parsep}{0pt}
			
			\item One
			\item two
			
		\end{itemize}
	\end{shtcutbox}
\end{center}

%***************************************************************************
\section{Box: Skill Refresher}
% Relies on ``sklbox'' in config file -- dark red color
%***************************************************************************
\begin{center}
	\begin{sklbox}{Skill Refresher}
		\textbf{Save}
		\\
		\begin{itemize}
			\setlength{\itemsep}{0pt}
			\setlength{\parskip}{0pt}
			\setlength{\parsep}{0pt}
			
			\item one
			\item two
			
		\end{itemize}
	\end{sklbox}
\end{center}

%***************************************************************************
\section{Box: Key Takeaways}
% Relies on ``tkwbox'' in config file -- green color
%***************************************************************************
\begin{center}
	\begin{tkwbox}{Key Take-Aways}
		\textbf{Save}
		\\
		\begin{itemize}
			\setlength{\itemsep}{0pt}
			\setlength{\parskip}{0pt}
			\setlength{\parsep}{0pt}
			
			\item one
			\item two
			
		\end{itemize}
	\end{tkwbox}
\end{center}

%***************************************************************************
\section{Enumerate Resume}
%***************************************************************************
\begin{enumerate}[resume]
	\item One
\end{enumerate}

%**********************************************************************
\section{Figure: Without Caption}
%**********************************************************************
% Use this for figures that need no caption

\begin{wrapfigure}{r}{0.4\textwidth}
	\centering
	\includegraphics[width=0.4\textwidth]{gfx/05-cake} 
\end{wrapfigure}

First paragraph.\blfootnote{Photo by lindsay Cotter on Unsplash}

%**********************************************************************
\section{Figure: Insert Into Enum List}
%**********************************************************************

\end{enumerate}

\begin{figure}[H]
\centering
\includegraphics[width=\maxwidth{.95\linewidth}]{gfx/ch08_fig3}
\caption{x}
\label{08:fig3}
\end{figure}

\begin{enumerate}[resume]

%**********************************************************************
\section{Figure: Wrap}
%**********************************************************************
\begin{wrapfigure}{O}{0.2\textwidth}
	\caption{} % No caption, wraps badly in very narrow space (does print fig number)
	% to not print a fig number use \caption*{}
	\label{03:fig02} 
	\centering
	\includegraphics[width=0.2\textwidth]{gfx/99-placeholder} 
\end{wrapfigure}

%***************************************************************************
\section{Text Table: Default For This Document}
%***************************************************************************
\begin{table}[H]
	\rowcolors{1}{}{tablerow} % zebra striping background
	{\small
		%\fontsize{8}{10} \selectfont %Replace small for special font size
		\begin{longtable}{L{1.0in}L{1.00in}} %Left-aligned, Max width: 4.25in
			\textbf{ColA} & \textbf{ColB} \endhead
			\hline
			A & B\\
			W & X\\
			\rowcolor{captionwhite}
			\caption{caption}
			\label{05:tab01}
		\end{longtable}
	} % End small
\end{table}


%***************************************************************************
\section{ToDo}
%***************************************************************************
% Create a ToDo note in the text. This also creates a new clickable TODO
% section in the ``structure'' box on the left side of the page.
%TODO This is a todo note.

%***************************************************************************
\section{PicPick Image Notes}
%***************************************************************************
To match the images in the Excel book, use these settings for text boxes:
Place a rectangle with rounded corners on the image.
Click the ``text'' tab for the object
Enter the text for the inside of the rectangle. Use mixed case and a period at the end of the sentence
Font: Tahoma 10pt (not bold)
Line Color: A43A37
Fill Color: FFF2CE
Background opacity: 100%
Arrow Style: 2


%**********************************************************************
\section{Chapter Checklist}
%**********************************************************************

To check each chapter, search for:
\fmtPopupButton to \fmtButton
\fmtRibbonButton to \fmtButton
\fmtWorksheetName to \fmtWorksheet
\fmtWorkbookName to \fmtWorksheet
\fmtCellLocation to \fmtLoc
\fmtPopupBox to \textit
Check that all file names include a hyphen, like ``CH2-yada.xlxs'' (Search for ``CH7'' since all of the file names start with the chapter number.)
\fmtRibbonGroup since that will signal a ribbon location
\fmtRibbonTab to see if this is a ribbon location
Remove all special formatting from text areas, they should only use italics
Row and Column identifiers to format those (like \textit{Row} $ 2 $), \textit{Column B}
Change ``highlight'' to ``select'' when selecting a range

%**********************************************************************
\section{Style Guide}
%**********************************************************************

\subsection{Text Formatting Codes}
I defined several format commands in ``Excel-config.tex''. Each of the commands start with ``fmt...'' so they are easy to find. By using these, I can re-format an item throughout the entire text by changing only the lines in the config file.

fmtButton: Anything clickable, and the labels for items in a dialog box
fmtKeystroke: Any key stroke
fmtLoc: Any cell location, including column or row name \fmtLoc{Row $ 1 $}
fmtNewExcel: Excel 365 information (use: fmtNewExcel{Excel 365})
fmtOldExcel: Excel 2016 information (use: fmtOldExcel{Excel 2016})
fmtTyping: Anything typed in
fmtWorksheet: Any workbook or worksheet or file name

% *** The following are obsolete
% CellLoc: Cell location (A1), column or row name (Column A, Row 1)
% Keystroke: Keystrokes (like: Enter)
% PopupBox: Popup box names
% PopupButton: Anything clickable in a popup box, like select lists
% RibbonButton: Any clickable item on the ribbon (like Find and Autosum)
% RibbonGroup: Ribbon Group
% RibbonTab: Ribbon Tab
% Typing: Typed-in Text, like formulas
% WorkbookName: Workbook (file) name
% WorksheetName: Worksheet name
% ExcelNew: Excel 365 information
% ExcelOld: Excel 2016 information


\subsection{Special Character Formats}
My tilde character: $\sim$

``Control'' is abbreviated ``Ctrl''

Locations in text but not a procedure, like $ A1 $:$ D1 $ (note, italics doesn't matter with a math formula). Note, leave the colon outside the math formula for better spacing.

The \fmtButton{Auto Fill Handle} should be formatted as a button.

In text, function names are in all caps with no other formatting, like SUM and COUNTA. In a procedure, they are all caps but formatted as typing or a button (whichever is appropriate).

Italicize:
  Excel error messages that are displayed, like \textit{\#DIV/$ 0 $} (Numbers in an error message must be enclosed in dollar signs).
    
  \textit{Backstage} is italicized, but buttons on the Backstage view are formatted as buttons in a procedure.
  
  The names of dialog boxes
  
  Functions, sheetname, etc. in text, not a procedural step - this includes Skill Refreher boxes (as an exception, keystrokes are always formatted with \fmtKeystroke)
  
  Row/column designations in text, not a procedural step. Enclose row number in in-line math symbols (dollar signs) outside the italics section: \textit{Row} $ 25 $.
  
  Color selectors, like \textit{Blue, Accent 1, Darker 25\%}

\subsection{Objectives/Takeaways}

Learning Objectives should be bullet-points that start with an action verb, like: Learn how to save workbooks. These are only used for Sections, nothing lesser

Key Takeaways are more verbose and briefly summarize the main elements of the section. These are used to bookend a Learning Objectives element.

\subsection{Ribbon Locations}
Click \fmtButton{Tab $ \Rightarrow $ Group $ \Rightarrow $ Button}.

Click \fmtButton{File $ \Rightarrow $ Print}.

(\fmtNewExcel{Excel 365}. The tab is named \textit{Table Design}, not \textit{Table Tools Design}.)

\subsection{Cross-Reference}
Cross-references should be placed in a footnote rather than in-line.
\footnote{See Chapter \ref{ch01:fundamentals}, \nameref{ch01:fundamentals}, page \pageref{ch01:fundamentals} for more information about Excel fundamentals.}
\footnote{See Chapter \ref{ch02:computations}, \nameref{ch02:computations}, page \pageref{ch02:computations} for more information about mathematical computations.}
\footnote{See Chapter \ref{ch03:formulas}, \nameref{ch03:formulas}, page \pageref{ch03:formulas} for more information about formulas.}
\footnote{See Chapter \ref{ch04:charts}, \nameref{ch04:charts}, page \pageref{ch04:charts} for more information about charts.}
\footnote{See Chapter \ref{ch05:tables}, \nameref{ch05:tables}, page \pageref{ch05:tables} for more information about tables.}
\footnote{See Chapter \ref{ch06:sheets}, \nameref{ch06:sheets}, page \pageref{ch06:sheets} for more information about using multiple sheets.}
\footnote{See Chapter \ref{ch07:data}, \nameref{ch07:data}, page \pageref{ch07:data} for more information about manipulating data.}
\footnote{See Chapter \ref{ch08:forecasting}, \nameref{ch08:forecasting}, page \pageref{ch08:forecasting} for more information about forecasting.}
\footnote{See Chapter \ref{ch09:topics}, \nameref{ch09:topics}, page \pageref{ch09:topics} for more information.}

\subsection{Labels}
Labels should be a chapter, colon, verbal desc (all lc with underscores): \label{03:title}
For figures, label should be ``ch'' + chapter number, underscore, ``fig'' + number: ch01_fig01
For tables, label should be chapter number + colon + ``tab'' + number: 03:tab01
Reference: Chapter \ref{ch09:topics}, \nameref{ch09:topics}, page \pageref{ch09:topics}

\subsection{Misc}
Use ``Click in \fmtLoc{A1}'' to specify a single cell, ``Select the range \fmtLoc{A1:C2}'' to select a range.

The text width in a normal line is 11.80737cm

Dates do not include an apostrophe: In the 1960s this happened...

Wrap all numbers in an in-line math block: $ 123 $, even if the number is also formatted for text entry: \fmtTyping{$ 123 $}

The word ``data'' is singular, thus: ``Data is analyzed...''

Insert chapter cross-refs in a footnote.

Clickable items in an Objectives or Takeaway box are italicized rather than using the \fmt... commands.


%**********************************************************************
\section{Obsolete Stuff}
%**********************************************************************
The following items were copied in from other books I've written but are not used here. I left them in so I can reuse them if I want.

%**************************************************************************
%\section{Margin Paragraph} 
%**************************************************************************
% Use to create a sidebar paragraph out in the margin
%\marginpar{Whatever.}

%**************************************************************************
%\section{Footnote Without Marker} 
%**************************************************************************
% Use to create a footnote without a numbering marker
%\blfootnote{Whatever.}

%***************************************************************************
%\section{Handy Regex Expressions}
%***************************************************************************
%All caps: \b([A-Z][A-Z]+)\b NOTE: enable case-sensitive search
%One or more numbers followed by a space: \d+\s
%One or more numbers followed by a period: \d+\.

%***************************************************************************
%\section{Text Table: Long}
%***************************************************************************
% instead of using a standard font size, like small, I can specify
% a point size using the following line. This is {font points} and 
% {base line stretch}.
% \fontsize{8}{10} \selectfont
%{\small
%\begin{longtable}{L{0.75in}L{1.5in}L{1.5in}} %Max width: 4.25in
%	\textbf{Command} & \textbf{Purpose} & \textbf{Use} \endhead
%	\hline \\
%	Margins & Sets the top, bottom, right, and left margin space for the printed document & 1. Click the Page Layout tab of the Ribbon.\newline2. Click the Margin button.\newline3. Click one of the preset margin options or click Custom Margins.\\
%	\caption{Purpose and Use for Page Setup Commands}
%\label{01:tab02}
%\end{longtable}
%}

%***************************************************************************
%\section{Text Table: Simple}
%***************************************************************************
%\begin{table}[H]
%	\centering
%	\begin{tabulary}{\linewidth}{LCCCC}
%		\hline
%		\multicolumn{5}{l}{\textbf{Do you support these propositions?}} \\
%		\hline
%		Prop & Strongly Support & Support & Do Not Support & Strongly Do Not Support  \\ 
%		\hline
%		100 & $\bigcirc$ & $\bigcirc$ & $\bigcirc$ & $\bigcirc$ \\ 
%		115 & $\bigcirc$ & $\bigcirc$ & $\bigcirc$ & $\bigcirc$ \\ 
%		220 & $\bigcirc$ & $\bigcirc$ & $\bigcirc$ & $\bigcirc$ \\ 
%		\hline
%	\end{tabulary} 
%	\caption{My great table.}
%	\label{03:tab01}
%\end{table}
