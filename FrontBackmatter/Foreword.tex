%************************************************
\chapter*{Foreword}\label{ch:foreword}
%************************************************

Microsoft Excel is a spreadsheet program developed by Microsoft for Windows, MacOS, Android, and iOS and is part of the Office suite of software. It features calculation, graphing tools, pivot tables, and a macro programming language called \textit{Visual Basic for Applications}. Excel is used widely for many financially-related activities, from simple quarterly forecasts to complete corporate annual reports. Excel is also used for typical information organization like contact lists and inventory tracking. Finally, Excel helps researchers perform statistical analysis tasks like variance analysis, chi-square testing, and charting complex data.

I have used Excel for business and personal use for more than $ 20 $ years. For the \textit{Cochise College Center for Lifelong Learning} class, I started with an ``open source'' book since it is available free of charge, and I could modify it to meet the objectives of this class.

\begin{quote}
	\textit{Beginning Excel} by Noreen Brown, Barbara Lave, Julie Romey, Mary Schatz, Diane Shingledecker. I found it at \textit{Open Oregon Educational Resources}, \url{https://ecampusontario.pressbooks.pub/beginningexcel/}
\end{quote}

I sincerely appreciate the original authors' work and their decision to publish their book under the Creative Commons so I could modify it for my class. The original book has six chapters, but I added three chapters with advanced skills to better fit my class and then reformatted the entire work into my voice.

This book includes information for both Excel $ 2016 $ and Excel $ 365 $. These two products are identical for the most part; however, a difference in functionality will be identified with one of two symbols. 

\begin{itemize}
	\item \fmtOldExcel{Excel 2016} Something only found in Excel $ 2016 $. 
	\item \fmtNewExcel{Excel 365} Something only found in Excel $ 365 $.
\end{itemize}

You will find references like this throughout the book: Click \fmtButton{Home $ \Rightarrow $ Cells $ \Rightarrow $ Clear}. This series means to look on the \textit{Home} tab, the \textit{Cells} group of commands, and click the \textit{Clear} button.

The book includes both background information and actionable steps to complete. Activity steps are printed with a distinctive background color, as illustrated below.

\begin{enumbox}
	\begin{enumerate}
	\item This is step one.
	\item This is step two.
	\item This is step three.
	\end{enumerate}
\end{enumbox}

While this book is helpful in its current form, I will continually update it based on my research and students' experiences. I hope that students can use this book to learn about Excel and that other instructors can adapt it for their classes.

\bigskip
\begin{flushright}
  \textemdash \; George Self
\end{flushright}

% Can add this later if I use anything from this book.
% \textit{How to Use Microsoft Excel: The Careers in Practice Series}, adapted by \textit{The Saylor Foundation} without attribution as requested by the work's original creator. It was downloaded from \url{https://resources.saylor.org/wwwresources/archived/site/textbooks/How%20to%20Use%20Microsoft%20Excel.pdf}
