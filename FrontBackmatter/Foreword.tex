%************************************************
\chapter*{Foreword}\label{ch:foreword}
%************************************************

Microsoft Excel is a spreadsheet program developed by Microsoft for Windows, MacOS, Android and iOS and is part of the Office suite of software. It features calculation, graphing tools, pivot tables, and a macro programming language called \textit{Visual Basic for Applications}. Excel is used widely for many financially-related activities from simple quarterly forecasts to full corporate annual reports. Excel is also used for common information organization like contact lists and inventory tracking. Finally, Excel helps researchers perform statistical analysis tasks like variance analysis, chi-square testing, and charting complex data.

I've used Excel for both business and personal use for more than 20 years. For the Cochise College Center for Lifelong Learning class, I started with an ``open source'' book since those are available free of charge and I could modify it to meet the objectives of this class. I found two books: 

\begin{itemize}
	\item \textit{Beginning Excel} by Noreen Brown, Barbara Lave, Julie Romey, Mary Schatz, Diane Shingledecker. I found it at \textit{Open Oregon Educational Resources}, \url{https://ecampusontario.pressbooks.pub/beginningexcel/}.

	\item \textit{How to Use Microsoft Excel: The Careers in Practice Series}, adapted by \textit{The Saylor Foundation} without attribution as requested by the work's original creator. It was downloaded from \url{https://resources.saylor.org/wwwresources/archived/site/textbooks/How%20to%20Use%20Microsoft%20Excel.pdf}
\end{itemize}

While the book is useful in its current form, I will continually update it based on emerging trends in research. It is my hope that students can use this book to learn about Excel and other instructors can adapt it for their own classes.

\bigskip
\begin{flushright}
  \textemdash \; George Self
\end{flushright}
