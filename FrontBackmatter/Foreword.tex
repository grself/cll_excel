%************************************************
\chapter*{Foreword}\label{ch:foreword}
%************************************************

Microsoft Excel is a spreadsheet program developed by Microsoft for Windows, MacOS, Android, and iOS and is part of the Office suite of software. It features calculation, graphing tools, pivot tables, and a macro programming language called \textit{Visual Basic for Applications}. Excel is used widely for many financially-related activities from simple quarterly forecasts to full corporate annual reports. Excel is also used for common information organization like contact lists and inventory tracking. Finally, Excel helps researchers perform statistical analysis tasks like variance analysis, chi-square testing, and charting complex data.

I've used Excel for both business and personal use for more than $ 20 $ years. For the \textit{Cochise College Center for Lifelong Learning} class, I started with an ``open source'' book since it is available free of charge and I could modify it to meet the objectives of this class.

\begin{quote}
	\textit{Beginning Excel} by Noreen Brown, Barbara Lave, Julie Romey, Mary Schatz, Diane Shingledecker. I found it at \textit{Open Oregon Educational Resources}, \url{https://ecampusontario.pressbooks.pub/beginningexcel/}
\end{quote}

I deeply appreciate the original authors' work and their decision to publish their book under the Creative Commons so I could modify it for my class. The original book has six chapters, but to better fit my class I added three chapters with advanced skills and then reformatted the entire work into my own voice.

This book includes information for both Excel $ 2016 $ and Excel $ 365 $. These two products are identical for the most part; however, when there is a difference in functionality that will be identified with one of two symbols. 

\begin{itemize}
	\item \fmtOldExcel{Excel 2016} This is something that is only found in Excel $ 2016 $. 
	\item \fmtNewExcel{Excel 365} This is something that is only found in Excel $ 365 $.
\end{itemize}

Throughout the book you will find references that look like this: Click \fmtButton{Home $ \Rightarrow $ Cells $ \Rightarrow $ Clear}. This means to look on the \textit{Home} tab, the \textit{Cells} group of commands, and click the \textit{Clear} button.

While this book is useful in its current form, I will continually update it based on my students' experiences and my own research. It is my hope that students can use this book to learn about Excel and other instructors can adapt it for their own classes.

\bigskip
\begin{flushright}
  \textemdash \; George Self
\end{flushright}

% Can add this later if I use anything from this book.
% \textit{How to Use Microsoft Excel: The Careers in Practice Series}, adapted by \textit{The Saylor Foundation} without attribution as requested by the work's original creator. It was downloaded from \url{https://resources.saylor.org/wwwresources/archived/site/textbooks/How%20to%20Use%20Microsoft%20Excel.pdf}
